% The table of contents outputs the sections and subsections that appear in your presentation, specified with the standard \section and \subsection commands. You may either display all sections and subsections on one slide with \tableofcontents, or display each section at a time on subsequent slides with \tableofcontents[pausesections]. The latter is useful if you want to step through each section and mention what you will discuss.
% Control the font size of sections and subsections in the table of contents
\setbeamerfont{section in toc}{size=\small} % Set section font size
\setbeamerfont{subsection in toc}{size=\tiny} % Set subsection font size

\begin{frame}
	\frametitle{Outline} % Slide title, remove this command for no title
	\tableofcontents % Output the table of contents (all sections on one slide)
	%\tableofcontents[pausesections] % Output the table of contents (break sections up across separate slides)
\end{frame}