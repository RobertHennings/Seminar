\begin{frame}
	\frametitle{Intro: Energy Commodities and Exchange Rates}
	% \begin{itemize}
	% 	{\item \small \textit{This has led some to suggest that an unidentified real factor may be causing persistent shifts in real equilibrium exchange rates.\footnote{Oil prices and the rise and fall of the US real exchange rate, R.A. Amano, S. van Norden, Journal of International Money and Finance 17 (1998) 299-316, p.301}}}
	% 	{\item  \small \textit{The link between nominal exchange rates and price differentials (reflecting the validity of purchasing power parity (PPP), which constitutes a part of the terms-of-trade channel) is characterized by several nonlinearities.\footnote{The relationship between oil prices and exchange rates: Revisiting theory and evidence, J. Beckmann, R. Czudaj, V Arora, Energy Economics 88 (2020), p.6}}}
	% \end{itemize}
	% Quote example
	\begin{quote}
		\small{This has led some to suggest that an unidentified real factor may be causing persistent shifts in real equilibrium exchange rates.}\\
		\small{--- R.A. Amano, S. van Norden}\footnote{Oil prices and the rise and fall of the US real exchange rate, R.A. Amano, S. van Norden, Journal of International Money and Finance 17 (1998) 299-316, p.301}
	\end{quote}
	\bigskip % Vertical whitespace
	\begin{quote}
		\small{This may in fact be the case or it is also possible that the relationship between exchange rates and oil shocks is non-linear and not being detected by a linear regression framework.}\\
		\small{--- S. A. Basher, A. A. Haug, P. Sadorsky}\footnote{The impact of oil shocks on exchange rates: A Markov-switching approach, S. A. Basher, A. A. Haug, P. Sadorsky, Energy Economics 54 (2016) 11–23, p.17}
	\end{quote}
	\bigskip % Vertical whitespace
	\begin{quote}
		\small{The long-run real exchange rate of these ‘commodity currencies’ is not constant (as would be implied by purchasing power parity-based models) but is time varying, being dependent on movements in the real price of commodity exports.}\\
		\small{--- P. Cashin, L. F. Cespedes, R. Sahay}\footnote{Commodity currencies and the real exchange rate, P. Cashin, L. F. Cespedes, R. Sahay, Journal of Development Economics 75 (2004) 239–268, p.239}
	\end{quote}
\end{frame}
%----------------------------------------------------------------------------------------
%	NEW SLIDE
%----------------------------------------------------------------------------------------
\begin{frame}
	\frametitle{Intro: The PPP puzzle and Commodity Currencies}

	\begin{figure}
		\includegraphics[width=0.8\linewidth]{chap_00_deviations_of_usd_spotrates_from_ppp_values.pdf}
		\caption{\tiny Figure: Monthly deviations of USD Spot Rate from PPP-values (in log terms) over the time: 1964 - 2025.\footnote{Own Illustration based on XYZ (2000), page 7 and data accessed 10.09.25.\\\tiny This puzzle concerns the finding of many researchers that the speed of mean reversion of real exchange rates is too slow to be consistent with PPP, which is the proposition that exchange rates are determined by movements in relative prices.}\label{fig:chap_00_deviations_of_usd_spotrates_from_ppp_values}}
	\end{figure}
\end{frame}