%----------------------------------------------------------------------------------------
%	NEW CHAPTER TITLE SLIDE
%----------------------------------------------------------------------------------------
\begin{frame}
	\frametitle{Chapter 2)}
	\begin{tikzpicture}[remember picture, overlay]
        \node[anchor=north east, xshift=-0.1cm, yshift=+0.1cm] at (current page.north east) {
            \includegraphics[width=1.7cm]{wiso_logo.png}
        };
		\node[anchor=center, xshift=5.5cm, yshift=-1.5cm] at (current page.center) {
            \tikz\node[opacity=0.15] {
                \includegraphics[width=\paperwidth, height=\paperheight, keepaspectratio]{siegel.pdf}
            };
        };
    \end{tikzpicture}
	% \begin{block}{} % Empty block title % Set background color (adjust 'blue!20' as needed)
    %     \centering
    %     {\large \textbf{Research Hypothesis}} % Slightly smaller text
    % \end{block}
	\begin{center}
        \begin{minipage}{0.6\textwidth} % Adjust the width of the block
            \begin{block}{} % Empty block title
                \centering
                {\large \textbf{Research Hypothesis}} % Slightly smaller text
            \end{block}
        \end{minipage}
    \end{center}
\end{frame}
%----------------------------------------------------------------------------------------
\subsection{Energy Commodity Price Shocks: The Pass-Through Effect and implications for Monetary Policy}
%----------------------------------------------------------------------------------------
%	NEW SLIDE
%----------------------------------------------------------------------------------------
\begin{frame}
	\frametitle{Energy Price Contributions to Inflation}
	% \begin{itemize}
	% 	\item Energy prices have been a significant driver of inflation in recent years, particularly due to geopolitical tensions and supply chain disruptions.
	% 	\item The volatility in energy markets has led to increased costs for transportation, manufacturing, and household energy consumption.
	% 	\item Central banks face challenges in managing inflation expectations while considering the transitory nature of energy price shocks.
	% \end{itemize}
	\begin{figure}
		\includegraphics[width=0.8\linewidth]{us_cpi_components_contribution.pdf}
		\caption{\tiny Figure: US CPI and its main components over the time: 2000 - 2025.\footnote{Own Illustration based on U.S. Bureau of Labor Statistics (2025), page 7 and data accessed 10.09.25.}}
	\end{figure}
\end{frame}
%----------------------------------------------------------------------------------------
%	NEW SLIDE
%----------------------------------------------------------------------------------------
\begin{frame}
	\frametitle{Energy Price Contributions to Inflation}
	% \begin{itemize}
	% 	\item Energy prices have been a significant driver of inflation in recent years, particularly due to geopolitical tensions and supply chain disruptions.
	% 	\item The volatility in energy markets has led to increased costs for transportation, manufacturing, and household energy consumption.
	% 	\item Central banks face challenges in managing inflation expectations while considering the transitory nature of energy price shocks.
	% \end{itemize}
	\begin{figure}
		\includegraphics[width=0.8\linewidth]{eu_area_cpi_components_contribution.pdf}
		\caption{\tiny Figure: EU Area CPI and its main components over the time: 2000 - 2025.\footnote{Own Illustration based on U.S. Bureau of Labor Statistics (2025), page 7 and data accessed 10.09.25.}}
	\end{figure}
\end{frame}
%----------------------------------------------------------------------------------------
\subsection{Energy Commodity Price Shocks: Economic Rationale}
%----------------------------------------------------------------------------------------
%	NEW SLIDE
%----------------------------------------------------------------------------------------
\begin{frame}
	\frametitle{Energy Commodity Price Shocks: Economic Rationale}
	% \begin{itemize}
	% 	\item Energy prices have been a significant driver of inflation in recent years, particularly due to geopolitical tensions and supply chain disruptions.
	% 	\item The volatility in energy markets has led to increased costs for transportation, manufacturing, and household energy consumption.
	% 	\item Central banks face challenges in managing inflation expectations while considering the transitory nature of energy price shocks.
	% \end{itemize}
	\begin{figure}
		\includegraphics[width=0.8\linewidth]{oil_gas_usd_index.pdf}
		\caption{\tiny Figure: USD Index, WTI and Natural Gas over the time: 2000 - 2025.\footnote{Own Illustration based on U.S. Bureau of Labor Statistics (2025), page 7 and data accessed 10.09.25.}}
	\end{figure}
\end{frame}
%----------------------------------------------------------------------------------------
\subsection{Formulated Research Hypothesis}
%----------------------------------------------------------------------------------------
%	NEW SLIDE
%----------------------------------------------------------------------------------------
\begin{frame}
	\frametitle{Formulated Research Hypothesis}
	\begin{block}{\underline{Main Research Hypothesis}}
		\centering
		\textit{``Exchange Rates and energy commodity prices are interconnected over several time frequencies and horizons, predominantly during times of (financial market) distress. Energy commodity price shocks primarily enter through the inflation dynamics channel, influencing both short-term price levels and long-term inflation expectations, thereby also affecting monetary policy decisions.''}
	\end{block}
	\begin{block}{Aditional Research Hypothesis I}
		\centering
		\textit{``The pass-through effect of energy commodity price shocks to overall inflation is asymmetric, non-linear and time-varying, with price increases having a more pronounced effect than price decreases.''}
	\end{block}
	\begin{block}{Aditional Research Hypothesis II}
		\centering
		\textit{``The pass-through effect intensified with growing financialization of energy commodity markets, leading to stronger correlations between energy prices, exchange rates, and inflation.''}
	\end{block}
\end{frame}