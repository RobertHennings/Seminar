%----------------------------------------------------------------------------------------
%	NEW CHAPTER TITLE SLIDE
%----------------------------------------------------------------------------------------
\begin{frame}
	\frametitle{Chapter 4)}
	\begin{tikzpicture}[remember picture, overlay]
        \node[anchor=north east, xshift=-0.1cm, yshift=+0.1cm] at (current page.north east) {
            \includegraphics[width=1.7cm]{wiso_logo.png}
        };
		\node[anchor=center, xshift=5.5cm, yshift=-1.5cm] at (current page.center) {
            \tikz\node[opacity=0.15] {
                \includegraphics[width=\paperwidth, height=\paperheight, keepaspectratio]{siegel.pdf}
            };
        };
    \end{tikzpicture}
	% \begin{block}{} % Empty block title % Set background color (adjust 'blue!20' as needed)
    %     \centering
    %     {\large \textbf{Theoretical Framework}} % Slightly smaller text
    % \end{block}
	\begin{center}
        \begin{minipage}{0.6\textwidth} % Adjust the width of the block
            \begin{block}{} % Empty block title
                \centering
                {\large \textbf{Conclusion and Discussion}} % Slightly smaller text
            \end{block}
        \end{minipage}
    \end{center}
\end{frame}
%----------------------------------------------------------------------------------------
\subsection{Seminar Project Summary}
%----------------------------------------------------------------------------------------
%	NEW SLIDE
%----------------------------------------------------------------------------------------
\begin{frame}
	\frametitle{Seminar Project Summary}

    \begin{block}{Main Research Hypothesis}
		Comparing different (configurations of) optimised clustering algorithms to identify (volatility driven) exchange rate regimes in the EUR/USD spot exchange rate yields promising results.
	\end{block}

    \begin{block}{Additional Research Hypothesis I}
		Incorporating commodity prices (WTI Crude Oil Prices and Natural Gas Prices/ their rolling volatilities) as external variables improves the regime identification results.
	\end{block}

    \begin{block}{Additional Research Hypothesis II}
		In comparison with the benchmark models, the clustering techniques seemed to be able to identify regimes more granular, but despite that were not able to yield more reliable/robust results for the standard UIP-relationship.
	\end{block}

\end{frame}
%----------------------------------------------------------------------------------------
\subsection{Seminar Project Limitations}
%----------------------------------------------------------------------------------------
%	NEW SLIDE
%----------------------------------------------------------------------------------------
\begin{frame}
	\frametitle{Seminar Project Limitations}

    \begin{block}{Evaluation Metrics}
		Only used one metric for clustering (Silhouette Score).
	\end{block}

    \begin{block}{Crisis Periods}
		May include more periods or find a suitable solution for overlapping crisis periods.
	\end{block}

    \begin{block}{Hyperparameter Tuning}
		Hyperparameter tuning did not finish in time for all model configurations.
	\end{block}
\end{frame}
%----------------------------------------------------------------------------------------
\subsection{Future Research}
%----------------------------------------------------------------------------------------
%	NEW SLIDE
%----------------------------------------------------------------------------------------
\begin{frame}
	\frametitle{Future Research - Possible Extensions}
    \begin{block}{Other Spot rate pairs}
		Extend the analysis to other spot exchange rate pairs, e.g. focus on the Commodity Currencies with a strong commodity linkage.
	\end{block}
    \begin{block}{Other commodities}
		Extend the analysis to also include other commodities, e.g. Natural Gas, Brent Oil, Gold, etc..
	\end{block}
    \begin{block}{Other macroeconomic/external variables}
		Extend the analysis by incorporating other macroeconomic variables, e.g. Interest Rates, Inflation Rates, etc..
	\end{block}
    \begin{block}{Other data frequencies}
		Extend the analysis by using intraday data to capture more granular dynamics.
	\end{block}
    \begin{block}{Other non-linear models}
		Apply Machine Learning techniques to enhance predictive capabilities and better capture non-linearities.
	\end{block}
\end{frame}