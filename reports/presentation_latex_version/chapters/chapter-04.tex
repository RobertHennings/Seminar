%----------------------------------------------------------------------------------------
%	NEW CHAPTER TITLE SLIDE
%----------------------------------------------------------------------------------------
\begin{frame}
	\frametitle{Chapter 3)}
	\begin{tikzpicture}[remember picture, overlay]
        \node[anchor=north east, xshift=-0.1cm, yshift=+0.1cm] at (current page.north east) {
            \includegraphics[width=1.7cm]{wiso_logo.png}
        };
		\node[anchor=center, xshift=5.5cm, yshift=-1.5cm] at (current page.center) {
            \tikz\node[opacity=0.15] {
                \includegraphics[width=\paperwidth, height=\paperheight, keepaspectratio]{siegel.pdf}
            };
        };
    \end{tikzpicture}
	% \begin{block}{} % Empty block title % Set background color (adjust 'blue!20' as needed)
    %     \centering
    %     {\large \textbf{Theoretical Framework}} % Slightly smaller text
    % \end{block}
	\begin{center}
        \begin{minipage}{0.6\textwidth} % Adjust the width of the block
            \begin{block}{} % Empty block title
                \centering
                {\large \textbf{Theoretical Framework}} % Slightly smaller text
            \end{block}
        \end{minipage}
    \end{center}
\end{frame}
%----------------------------------------------------------------------------------------
\subsection{Impact of Inflation on Measurements: What are prices and how are they measured?}
%----------------------------------------------------------------------------------------
%	NEW SLIDE
%----------------------------------------------------------------------------------------
\begin{frame}
	\frametitle{Definitions - prices and measurements}
    
	\hypertarget{fig:chap_04_theo_framework_prices_measures}{}
	\begin{figure}
		\includegraphics[width=1.0\linewidth]{chap_04_theo_framework_prices_measures.pdf}
		\caption{\tiny Figure: Schematic overview of prices and measurements of various exchange rate types.\footnote{Own Illustration based on \cite{rosenberg_2003_exchange_rate_determination}, page 8, exchange rates in natural logarithm (geometric averages) and data definitions from \cite{bis_2025_effective_exchange_rates}.}\label{fig:chap_04_theo_framework_prices_measures}}
	\end{figure}
\end{frame}
%----------------------------------------------------------------------------------------
\subsection{A simple model of exchange rates and commodity prices}
%----------------------------------------------------------------------------------------
%	NEW SLIDE
%----------------------------------------------------------------------------------------
\begin{frame}
	\frametitle{A simple model of exchange rates and commodity prices}
    
	\hypertarget{fig:chap_04_theo_framework_simple_model}{}
	\begin{figure}
		\includegraphics[width=1.0\linewidth]{chap_04_theo_framework_simple_model.pdf}
		\caption{\tiny Figure: Main theoretical framework for the seminar project analysis, inflation pass-through effect of energy commodity prices.\footnote{Own Illustration based on \cite{oberndorfer_2020_oil_prices_exchange_rates}, Figure 3, page 3.}\label{fig:chap_04_theo_framework_simple_model}}
	\end{figure}
\end{frame}
%----------------------------------------------------------------------------------------
\subsection{Theoretical Framework}
%----------------------------------------------------------------------------------------
%	NEW SLIDE
%----------------------------------------------------------------------------------------
\begin{frame}
	\frametitle{Theoretical Framework (I)}
    
	\hypertarget{fig:chap_04_theo_framework_theoretical_framework_I}{}
	\begin{figure}
		\includegraphics[width=1.0\linewidth]{chap_04_theo_framework_theoretical_framework_I.pdf}
		\caption{\tiny Figure: Main theoretical framework for the seminar project analysis, main variables and benchmark models used.\footnote{Own Illustration.}\label{fig:chap_04_theo_framework_theoretical_framework_I}}
	\end{figure}
\end{frame}
%----------------------------------------------------------------------------------------
%	NEW SLIDE
%----------------------------------------------------------------------------------------
\begin{frame}
	\frametitle{Theoretical Framework (II)}
    
	\hypertarget{fig:chap_04_theo_framework_theoretical_framework_II}{}
	\begin{figure}
		\includegraphics[width=1.0\linewidth]{chap_04_theo_framework_theoretical_framework_II.pdf}
		\caption{\tiny Figure: Main theoretical framework for the seminar project analysis, main algorithms used.\footnote{Own Illustration based on formulas taken from \cite{reitz_2025_applied_econometrics_fx}, section ”Modeling Nonlinearities I: Markov-Switching”, page 7-15 and algorithms from \cite{scikit_learn_1_7_2} and implemented in \cite{python_org}.}\label{fig:chap_04_theo_framework_theoretical_framework_II}}
	\end{figure}
\end{frame}
%----------------------------------------------------------------------------------------
%	NEW SLIDE
%----------------------------------------------------------------------------------------
\begin{frame}
	\frametitle{Theoretical Framework (III)}
    
	\hypertarget{fig:chap_04_theo_framework_theoretical_framework_III}{}
	\begin{figure}
		\includegraphics[width=1.0\linewidth]{chap_04_theo_framework_theoretical_framework_III.pdf}
		\caption{\tiny Figure: Main theoretical framework for the seminar project analysis, main evaluation metrics and crisis periods used.\footnote{Own Illustration based on data taken from \cite{fred_2025_federal_reserve_data}, last accessed 24.10.25.}\label{fig:chap_04_theo_framework_theoretical_framework_III}}
	\end{figure}
\end{frame}
%----------------------------------------------------------------------------------------
%	NEW SLIDE
%----------------------------------------------------------------------------------------
\begin{frame}
	\frametitle{Theoretical Framework (IV)}
    
	\hypertarget{fig:chap_04_theo_framework_theoretical_framework_IV}{}
	\begin{figure}
		\includegraphics[width=1.0\linewidth]{chap_04_theo_framework_theoretical_framework_IV.pdf}
		\caption{\tiny Figure: Main theoretical framework for the seminar project analysis, testing the regime dependent UIP-relationship.\footnote{Own Illustration based on formulas taken from \cite{reitz_2025_applied_econometrics_fx}, section ”Modeling Nonlinearities I: Markov-Switching”, page 7-15.}\label{fig:chap_04_theo_framework_theoretical_framework_IV}}
	\end{figure}
\end{frame}