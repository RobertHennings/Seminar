%------------------------------------------------------------------------------------
%	                   OTHER DOCUMENT CONFIGURATIONS
%------------------------------------------------------------------------------------
\documentclass[
	11pt, % Set the default font size, options include: 8pt, 9pt, 10pt, 11pt, 12pt, 14pt, 17pt, 20pt
	%t, % Uncomment to vertically align all slide content to the top of the slide, rather than the default centered
	%aspectratio=169, % Uncomment to set the aspect ratio to a 16:9 ratio which matches the aspect ratio of 1080p and 4K screens and projectors
]{beamer}

\graphicspath{{figures/}{./}} % Specifies where to look for included images (trailing slash required)
%%%%%%%%%%%%%%%%%%%%%%%%%%%%%%%%%%%%%%%%%
% Beamer Presentation
% LaTeX Template
% Version 2.0 (March 8, 2022)
%
% This template originates from:
% https://www.LaTeXTemplates.com
%
% Author:
% Vel (vel@latextemplates.com)
%
% License:
% CC BY-NC-SA 4.0 (https://creativecommons.org/licenses/by-nc-sa/4.0/)
%
%%%%%%%%%%%%%%%%%%%%%%%%%%%%%%%%%%%%%%%%%
%------------------------------------------------------------------------------------
%	                  LOAD .STY FILE
%------------------------------------------------------------------------------------
\usepackage{main}
%------------------------------------------------------------------------------------
%	                  REFERENCES
%------------------------------------------------------------------------------------
\usepackage[backend=bibtex, style=alphabetic]{biblatex}
% \usepackage[backend=biber, style=alphabetic]{biblatex}
% \usepackage[backend=biber, style=alphabetic, sorting=ynt]{biblatex}
\addbibresource{references.bib}
\defbibheading{subbibliography}[\bibname]{%
  \vspace{1ex}\noindent\textbf{#1}\par\smallskip%
}
\renewcommand*{\bibfont}{\tiny}
%------------------------------------------------------------------------------------
%	                  PREAMBLE
%------------------------------------------------------------------------------------
% \input(preamble) % Load the preamble file which contains custom commands and settings
%----------------------------------------------------------------------------------------
%	PRESENTATION INFORMATION
%----------------------------------------------------------------------------------------
\title[Energy Commodities and the EUR/USD]{Identifying (volatility-)regimes in the the EUR/USD spot exchange rate using clustering algorithms: \\ An Oil and Gas Perspective on Parity Conditions.} % The part in the [] is the shorttitle displayed in the middle bottom section of every slide
\subtitle{\tiny Seminar in Applied Financial Economics: Applied Econometrics of FX Markets - Prof. Dr. Reitz} % Presentation subtitle, remove this command if a subtitle isn't required

\author[Josef Fella and Robert Hennings]{Josef Fella and Robert Hennings} % Presenter name(s), the optional parameter can contain a shortened version to appear on the bottom of every slide, while the main parameter will appear on the title slide

\institute[CAU]{Christian Albrechts University of Kiel \\ \smallskip \textit{josef.fella@stu.uni-kiel.de and robert.hennings@stu.uni-kiel.de} \\ \smallskip \textit{GitHub:} \url{https://github.com/RobertHennings/Seminar}} % Your institution, the optional parameter can be used for the institution shorthand and will appear on the bottom of every slide after author names, while the required parameter is used on the title slide and can include your email address or additional information on separate lines

% \date[\today]{International Symposium of Explorers \\ \today} % Presentation date or conference/meeting name, the optional parameter can contain a shortened version to appear on the bottom of every slide, while the required parameter value is output to the title slide
% \date[\today]{\today} % Presentation date or conference/meeting name, the optional parameter can contain a shortened version to appear on the bottom of every slide, while the required parameter value is output to the title slide
\date[\today]{\small Kiel - 14.11.2025}
%----------------------------------------------------------------------------------------
\begin{document}
%----------------------------------------------------------------------------------------
%	TITLE SLIDE
%----------------------------------------------------------------------------------------
\begin{frame}[plain]
    \titlepage{}
    \begin{tikzpicture}[remember picture, overlay]
        \node[anchor=north east, xshift=-0.1cm, yshift=+0.1cm] at (current page.north east) {
            \includegraphics[width=1.7cm]{wiso_logo.png}
        };
		% Add the .pdf graph as a lightly shaded background
        \node[anchor=center, xshift=5.5cm, yshift=-1.5cm] at (current page.center) {
            \tikz\node[opacity=0.15] {
                \includegraphics[width=\paperwidth, height=\paperheight, keepaspectratio]{siegel.pdf}
            };
        };
    \end{tikzpicture}
\end{frame}

% \begin{frame}
% 	\titlepage % Output the title slide, automatically created using the text entered in the PRESENTATION INFORMATION block above
% \end{frame}
%----------------------------------------------------------------------------------------
%	TABLE OF CONTENTS SLIDE
%----------------------------------------------------------------------------------------
% The table of contents outputs the sections and subsections that appear in your presentation, specified with the standard \section and \subsection commands. You may either display all sections and subsections on one slide with \tableofcontents, or display each section at a time on subsequent slides with \tableofcontents[pausesections]. The latter is useful if you want to step through each section and mention what you will discuss.
% Control the font size of sections and subsections in the table of contents
\setbeamerfont{section in toc}{size=\small} % Set section font size
\setbeamerfont{subsection in toc}{size=\tiny} % Set subsection font size

\begin{frame}
	\frametitle{Outline} % Slide title, remove this command for no title
	\tableofcontents % Output the table of contents (all sections on one slide)
	%\tableofcontents[pausesections] % Output the table of contents (break sections up across separate slides)
\end{frame}
%----------------------------------------------------------------------------------------
%	PRESENTATION BODY SLIDES
%----------------------------------------------------------------------------------------

%----------------------------------------------------------------------------------------
%	00 - Intro
%----------------------------------------------------------------------------------------
%----------------------------------------------------------------------------------------
%	NEW SLIDE
%----------------------------------------------------------------------------------------
\begin{frame}
    \frametitle{Intro: Energy Commodities and Exchange Rates}
    \begin{quote}
        \small{This has led some to suggest that an unidentified real factor may be causing persistent shifts in real equilibrium exchange rates.}\\
        \small{--- R.A. Amano, S. van Norden}\footnote{\tiny \cite{amano_1998_oil_prices_exchange_rate}, p.301}
    \end{quote}
    \medskip % Reduced vertical space
    \begin{quote}
        \small{This may in fact be the case or it is also possible that the relationship between exchange rates and oil shocks is non-linear and not being detected by a linear regression framework.}\\
        \small{--- S. A. Basher, A. A. Haug, P. Sadorsky}\footnote{\tiny \cite{basher_2016_oil_shocks_exchange_rates}, p.17}
    \end{quote}
    \medskip

    \begin{quote}
        \small{The long-run real exchange rate of these `commodity currencies' is not constant but is time varying, being dependent on movements in the real price of commodity exports.}\\
        \small{--- P. Cashin, L. F. Cespedes, R. Sahay}\footnote{\tiny \cite{cashin2004}, p.239}
    \end{quote}
\end{frame}
%----------------------------------------------------------------------------------------
%	NEW SLIDE
%----------------------------------------------------------------------------------------
\begin{frame}
	\frametitle{Intro: The PPP puzzle\footnote{This puzzle concerns the finding of many researchers that the speed of mean reversion of real exchange rates is too slow to be consistent with PPP, which is the proposition that exchange rates are determined by movements in relative prices.} and Commodity Currencies}
	\hypertarget{fig:chap_00_deviations_of_usd_spotrates_from_ppp_values}{}
	\begin{figure}
		\href{https://roberthennings.github.io/Seminar/chap_00_deviations_of_usd_spotrates_from_ppp_values.html}{\includegraphics[width=0.7\linewidth]{chap_00_deviations_of_usd_spotrates_from_ppp_values.pdf}}
		\caption{\tiny Figure: Monthly deviations of USD Spot Rate from PPP-values (in log terms) over the time: 1964 - 2025.\footnote{Own Illustration based on \cite{reitz_2025_applied_econometrics_fx}, section "Modeling Trends: Unit Roots in Time Series", page 18/18 and data taken from \cite{bis_2025_effective_exchange_rates}, last accessed 24.10.25, own calculations.}\label{fig:chap_00_deviations_of_usd_spotrates_from_ppp_values}}
	\end{figure}
\end{frame}
%----------------------------------------------------------------------------------------
%	01 - Modern Commodities Markets - The current state
%----------------------------------------------------------------------------------------
% \section{Modern Energy Commodities Markets - The current state} % Sections are added in order to organize your presentation into discrete blocks, all sections and subsections are automatically output to the table of contents as an overview of the talk but NOT output in the presentation as separate slides
% %----------------------------------------------------------------------------------------
%	NEW CHAPTER TITLE SLIDE
%----------------------------------------------------------------------------------------
\begin{frame}
	\frametitle{Chapter 1)}
	\begin{tikzpicture}[remember picture, overlay]
        \node[anchor=north east, xshift=-0.1cm, yshift=+0.1cm] at (current page.north east) {
            \includegraphics[width=1.7cm]{wiso_logo.png}
        };
		\node[anchor=center, xshift=5.5cm, yshift=-1.5cm] at (current page.center) {
            \tikz\node[opacity=0.15] {
                \includegraphics[width=\paperwidth, height=\paperheight, keepaspectratio]{siegel.pdf}
            };
        };
    \end{tikzpicture}
	% \begin{block}{} % Empty block title % Set background color (adjust 'blue!20' as needed)
    %     \centering
    %     {\large \textbf{Modern Energy Commodity Markets - The current state}} % Slightly smaller text
    % \end{block}
	\begin{center}
        \begin{minipage}{0.6\textwidth} % Adjust the width of the block
            \begin{block}{} % Empty block title
                \centering
                {\large \textbf{Modern Energy Commodity Markets \\ The current state}} % Slightly smaller text
            \end{block}
        \end{minipage}
    \end{center}
\end{frame}
%----------------------------------------------------------------------------------------
\subsection{Physical Markets: Global Oil and Gas Production and Consumption over time}
%----------------------------------------------------------------------------------------
%	NEW SLIDE
%----------------------------------------------------------------------------------------
\begin{frame}
	\frametitle{Oil: Global Production and Consumption over time}
	\hypertarget{fig:chap_01_yearly_oil_consumption_production_combined_graph}{} % Added target for figure reference
	\begin{figure}
		\includegraphics[width=0.8\linewidth]{chap_01_yearly_oil_consumption_production_combined_graph.pdf}
		\caption{\tiny Figure: Yearly oil consumption and production by country over the time: 1965 - 2024 (in terawatt-hours).\footnote{Own Illustration based on XYZ (2000), page 7 and data taken from \cite{energy_institute_2025_oil_consumption}, last accessed 24.10.25.}\label{fig:chap_01_yearly_oil_consumption_production_combined_graph}}
	\end{figure}
\end{frame}
%----------------------------------------------------------------------------------------
%	NEW SLIDE
%----------------------------------------------------------------------------------------
\begin{frame}
	\frametitle{Gas: Global Production and Consumption over time}
	\hypertarget{fig:chap_01_yearly_gas_consumption_production_combined_graph}{} % Added target for figure reference
	\begin{figure}
		\includegraphics[width=0.8\linewidth]{chap_01_yearly_gas_consumption_production_combined_graph.pdf}
		\caption{\tiny Figure: Yearly gas consumption and production by country over the time: 1965 - 2024 (in terawatt-hours).\footnote{Own Illustration based on XYZ (2000), page 7 and data taken from \cite{energy_institute_2025_gas_consumption}, last accessed 24.10.25.}\label{fig:chap_01_yearly_gas_consumption_production_combined_graph}}
	\end{figure}
\end{frame}
%----------------------------------------------------------------------------------------
\subsection{Financial Markets: Global Oil and Gas Open Interest over time}
%----------------------------------------------------------------------------------------
%	NEW SLIDE
%----------------------------------------------------------------------------------------
\begin{frame}
	\frametitle{Financial Markets: Oil and Gas OI over time}
	\hypertarget{fig:chap_01_weekly_open_interest_oil_gas_combined_graph}{} % Added target for figure reference
	\begin{figure}
		\includegraphics[width=0.8\linewidth]{chap_01_weekly_open_interest_oil_gas_combined_graph.pdf}
		\caption{\tiny Figure: Weekly open interest of oil and gas products over the time: 2000 - 2024\footnote{Own Illustration based on XYZ (2000), page 7 and data taken from \cite{cftc_2025_commitments_of_traders}, last accessed 24.10.25.}\label{fig:chap_01_weekly_open_interest_oil_gas_combined_graph}}
	\end{figure}
\end{frame}
%----------------------------------------------------------------------------------------
%	02 - Research Hypothesis
%----------------------------------------------------------------------------------------
\section{Research Hypothesis} % Sections are added in order to organize your presentation into discrete blocks, all sections and subsections are automatically output to the table of contents as an overview of the talk but NOT output in the presentation as separate slides
%----------------------------------------------------------------------------------------
%	NEW CHAPTER TITLE SLIDE
%----------------------------------------------------------------------------------------
\begin{frame}
	\frametitle{Chapter 2)}
	\begin{tikzpicture}[remember picture, overlay]
        \node[anchor=north east, xshift=-0.1cm, yshift=+0.1cm] at (current page.north east) {
            \includegraphics[width=1.7cm]{wiso_logo.png}
        };
		\node[anchor=center, xshift=5.5cm, yshift=-1.5cm] at (current page.center) {
            \tikz\node[opacity=0.15] {
                \includegraphics[width=\paperwidth, height=\paperheight, keepaspectratio]{siegel.pdf}
            };
        };
    \end{tikzpicture}
	% \begin{block}{} % Empty block title % Set background color (adjust 'blue!20' as needed)
    %     \centering
    %     {\large \textbf{Research Hypothesis}} % Slightly smaller text
    % \end{block}
	\begin{center}
        \begin{minipage}{0.6\textwidth} % Adjust the width of the block
            \begin{block}{} % Empty block title
                \centering
                {\large \textbf{Research Hypothesis}} % Slightly smaller text
            \end{block}
        \end{minipage}
    \end{center}
\end{frame}
%----------------------------------------------------------------------------------------
\subsection{Energy Commodity Price Shocks: The Pass-Through Effect and implications for Monetary Policy}
%----------------------------------------------------------------------------------------
%	NEW SLIDE
%----------------------------------------------------------------------------------------
\begin{frame}
	\frametitle{Energy Price Contributions to Inflation}
	% \begin{itemize}
	% 	\item Energy prices have been a significant driver of inflation in recent years, particularly due to geopolitical tensions and supply chain disruptions.
	% 	\item The volatility in energy markets has led to increased costs for transportation, manufacturing, and household energy consumption.
	% 	\item Central banks face challenges in managing inflation expectations while considering the transitory nature of energy price shocks.
	% \end{itemize}
	\begin{figure}
		\includegraphics[width=0.8\linewidth]{us_cpi_components_contribution.pdf}
		\caption{\tiny Figure: US CPI and its main components over the time: 2000 - 2025.\footnote{Own Illustration based on U.S. Bureau of Labor Statistics (2025), page 7 and data accessed 10.09.25.}}
	\end{figure}
\end{frame}
%----------------------------------------------------------------------------------------
%	NEW SLIDE
%----------------------------------------------------------------------------------------
\begin{frame}
	\frametitle{Energy Price Contributions to Inflation}
	% \begin{itemize}
	% 	\item Energy prices have been a significant driver of inflation in recent years, particularly due to geopolitical tensions and supply chain disruptions.
	% 	\item The volatility in energy markets has led to increased costs for transportation, manufacturing, and household energy consumption.
	% 	\item Central banks face challenges in managing inflation expectations while considering the transitory nature of energy price shocks.
	% \end{itemize}
	\begin{figure}
		\includegraphics[width=0.8\linewidth]{eu_area_cpi_components_contribution.pdf}
		\caption{\tiny Figure: EU Area CPI and its main components over the time: 2000 - 2025.\footnote{Own Illustration based on U.S. Bureau of Labor Statistics (2025), page 7 and data accessed 10.09.25.}}
	\end{figure}
\end{frame}
%----------------------------------------------------------------------------------------
\subsection{Energy Commodity Price Shocks: Economic Rationale}
%----------------------------------------------------------------------------------------
%	NEW SLIDE
%----------------------------------------------------------------------------------------
\begin{frame}
	\frametitle{Energy Commodity Price Shocks: Economic Rationale}
	% \begin{itemize}
	% 	\item Energy prices have been a significant driver of inflation in recent years, particularly due to geopolitical tensions and supply chain disruptions.
	% 	\item The volatility in energy markets has led to increased costs for transportation, manufacturing, and household energy consumption.
	% 	\item Central banks face challenges in managing inflation expectations while considering the transitory nature of energy price shocks.
	% \end{itemize}
	\begin{figure}
		\includegraphics[width=0.8\linewidth]{oil_gas_usd_index.pdf}
		\caption{\tiny Figure: USD Index, WTI and Natural Gas over the time: 2000 - 2025.\footnote{Own Illustration based on U.S. Bureau of Labor Statistics (2025), page 7 and data accessed 10.09.25.}}
	\end{figure}
\end{frame}
%----------------------------------------------------------------------------------------
\subsection{Formulated Research Hypothesis}
%----------------------------------------------------------------------------------------
%	NEW SLIDE
%----------------------------------------------------------------------------------------
\begin{frame}
	\frametitle{Formulated Research Hypothesis}
	\begin{block}{\underline{Main Research Hypothesis}}
		\centering
		\textit{``Exchange Rates and energy commodity prices are interconnected over several time frequencies and horizons, predominantly during times of (financial market) distress. Energy commodity price shocks primarily enter through the inflation dynamics channel, influencing both short-term price levels and long-term inflation expectations, thereby also affecting monetary policy decisions.''}
	\end{block}
	\begin{block}{Aditional Research Hypothesis I}
		\centering
		\textit{``The pass-through effect of energy commodity price shocks to overall inflation is asymmetric, non-linear and time-varying, with price increases having a more pronounced effect than price decreases.''}
	\end{block}
	\begin{block}{Aditional Research Hypothesis II}
		\centering
		\textit{``The pass-through effect intensified with growing financialization of energy commodity markets, leading to stronger correlations between energy prices, exchange rates, and inflation.''}
	\end{block}
\end{frame}
%----------------------------------------------------------------------------------------
%	03 - Literature Review
%----------------------------------------------------------------------------------------
\section{Literature Review} % Sections are added in order to organize your presentation into discrete blocks, all sections and subsections are automatically output to the table of contents as an overview of the talk but NOT output in the presentation as separate slides
%----------------------------------------------------------------------------------------
%	NEW CHAPTER TITLE SLIDE
%----------------------------------------------------------------------------------------
\begin{frame}
	\frametitle{Chapter 2)}
	\begin{tikzpicture}[remember picture, overlay]
        \node[anchor=north east, xshift=-0.1cm, yshift=+0.1cm] at (current page.north east) {
            \includegraphics[width=1.7cm]{wiso_logo.png}
        };
		\node[anchor=center, xshift=5.5cm, yshift=-1.5cm] at (current page.center) {
            \tikz\node[opacity=0.15] {
                \includegraphics[width=\paperwidth, height=\paperheight, keepaspectratio]{siegel.pdf}
            };
        };
    \end{tikzpicture}
	% \begin{block}{} % Empty block title % Set background color (adjust 'blue!20' as needed)
    %     \centering
    %     {\large \textbf{Literature Review}} % Slightly smaller text
    % \end{block}
	\begin{center}
        \begin{minipage}{0.6\textwidth} % Adjust the width of the block
            \begin{block}{} % Empty block title
                \centering
                {\large \textbf{Literature Review}} % Slightly smaller text
            \end{block}
        \end{minipage}
    \end{center}
\end{frame}
%----------------------------------------------------------------------------------------
\subsection{Systematic Literature Overview: Main Approaches}
%----------------------------------------------------------------------------------------
%	NEW SLIDE
%----------------------------------------------------------------------------------------
\begin{frame}
	\frametitle{Systematic Literature Overview: Main Approaches}
	
    \hypertarget{fig:chap_03_literature_systematic_overview}{}
    \begin{figure}
		\includegraphics[width=1.0\linewidth]{chap_03_literature_systematic_overview.pdf}
		\caption{\tiny Figure: Systematic Overview about main theoretical approaches.\footnote{Own Illustration based on \cite{oberndorfer_2020_oil_prices_exchange_rates}, Figure 5, page 5.}\label{fig:chap_03_literature_systematic_overview}}
    \end{figure}
\end{frame}
%----------------------------------------------------------------------------------------
%	04 - Theoretical Framework
%----------------------------------------------------------------------------------------
\section{Theoretical Framework} % Sections are added in order to organize your presentation into discrete blocks, all sections and subsections are automatically output to the table of contents as an overview of the talk but NOT output in the presentation as separate slides
%----------------------------------------------------------------------------------------
%	NEW CHAPTER TITLE SLIDE
%----------------------------------------------------------------------------------------
\begin{frame}
	\frametitle{Chapter 2)}
	\begin{tikzpicture}[remember picture, overlay]
        \node[anchor=north east, xshift=-0.1cm, yshift=+0.1cm] at (current page.north east) {
            \includegraphics[width=1.7cm]{wiso_logo.png}
        };
		\node[anchor=center, xshift=5.5cm, yshift=-1.5cm] at (current page.center) {
            \tikz\node[opacity=0.15] {
                \includegraphics[width=\paperwidth, height=\paperheight, keepaspectratio]{siegel.pdf}
            };
        };
    \end{tikzpicture}
	% \begin{block}{} % Empty block title % Set background color (adjust 'blue!20' as needed)
    %     \centering
    %     {\large \textbf{Theoretical Framework}} % Slightly smaller text
    % \end{block}
	\begin{center}
        \begin{minipage}{0.6\textwidth} % Adjust the width of the block
            \begin{block}{} % Empty block title
                \centering
                {\large \textbf{Theoretical Framework}} % Slightly smaller text
            \end{block}
        \end{minipage}
    \end{center}
\end{frame}
%----------------------------------------------------------------------------------------
\subsection{Impact of Inflation on Measurements: What are prices and how are they measured?}
%----------------------------------------------------------------------------------------
\subsection{A simple model of exchange rates and commodity prices}
%----------------------------------------------------------------------------------------
%	NEW SLIDE
%----------------------------------------------------------------------------------------
\begin{frame}
	\frametitle{A simple model of exchange rates and commodity prices}
    
	\hypertarget{fig:chap_04_theo_framework_simple_model}{}
	\begin{figure}
		\includegraphics[width=1.0\linewidth]{chap_04_theo_framework_simple_model.pdf}
		\caption{\tiny Figure 4: Main theoretical framework for the seminar project analysis, inflation pass-through effect of energy commodity prices.\footnote{Own Illustration based on Figure 3, page 3, \cite{oberndorfer_2020_oil_prices_exchange_rates}.}\label{fig:chap_04_theo_framework_simple_model}}
	\end{figure}
\end{frame}
%----------------------------------------------------------------------------------------
\subsection{Theoretical Framework}
%----------------------------------------------------------------------------------------
%	NEW SLIDE
%----------------------------------------------------------------------------------------
\begin{frame}
	\frametitle{Theoretical Framework}
    
	\hypertarget{fig:chap_04_theo_framework_theoretical_framework_full}{}
	\begin{figure}
		\includegraphics[width=1.0\linewidth]{chap_04_theo_framework_theoretical_framework_full.pdf}
		\caption{\tiny Figure 5: Main theoretical framework for the seminar project analysis, key project steps.\footnote{Own Illustration with formulas taken from \cite{reitz_2025_applied_econometrics_fx}.}\label{fig:chap_04_theo_framework_theoretical_framework_full}}
	\end{figure}
\end{frame}
%----------------------------------------------------------------------------------------
%	05 - Methodology Overview
%----------------------------------------------------------------------------------------
% \section{Methodology Overview} % Sections are added in order to organize your presentation into discrete blocks, all sections and subsections are automatically output to the table of contents as an overview of the talk but NOT output in the presentation as separate slides
% %----------------------------------------------------------------------------------------
%	NEW CHAPTER TITLE SLIDE
%----------------------------------------------------------------------------------------
\begin{frame}
	\frametitle{Chapter 5)}
	\begin{tikzpicture}[remember picture, overlay]
        \node[anchor=north east, xshift=-0.1cm, yshift=+0.1cm] at (current page.north east) {
            \includegraphics[width=1.7cm]{wiso_logo.png}
        };
		\node[anchor=center, xshift=5.5cm, yshift=-1.5cm] at (current page.center) {
            \tikz\node[opacity=0.15] {
                \includegraphics[width=\paperwidth, height=\paperheight, keepaspectratio]{siegel.pdf}
            };
        };
    \end{tikzpicture}
	% \begin{block}{} % Empty block title % Set background color (adjust 'blue!20' as needed)
    %     \centering
    %     {\large \textbf{Theoretical Framework}} % Slightly smaller text
    % \end{block}
	\begin{center}
        \begin{minipage}{0.6\textwidth} % Adjust the width of the block
            \begin{block}{} % Empty block title
                \centering
                {\large \textbf{Methodology Overview}} % Slightly smaller text
            \end{block}
        \end{minipage}
    \end{center}
\end{frame}
%----------------------------------------------------------------------------------------
\subsection{Systematic Methodology Overview: Linear vs. Non-linear approaches}
%----------------------------------------------------------------------------------------
%	NEW SLIDE
%----------------------------------------------------------------------------------------
\begin{frame}
	\frametitle{Systematic Methodology Overview: Linear vs. Non-linear approaches}

\end{frame}
%----------------------------------------------------------------------------------------
\subsection{(Financial) Market Distress: Important periods and their characteristics}
%----------------------------------------------------------------------------------------
%	NEW SLIDE
%----------------------------------------------------------------------------------------
\begin{frame}
	\frametitle{(Financial) Market Distress: Important periods and their characteristics}
    % Here a list with the main periods of distress and maybe also a plot showing these periods like the US recessions
    % the Iranian revolution in 1979, the Iraqi invasion of Kuwait in 1990, the Asian financial crisis in 1997/98, production target cuts by OPEC in 1999, the dot-com bubble crisis in 2000, the terrorist at- tack on the World Trade Center in 2001, the Iraq War in 2003, the Global Financial Crisis in 2007/08, OPEC oil production cuts in 2009, and the European sovereign debt crisis starting in 2009
\end{frame}
%----------------------------------------------------------------------------------------
%	06 - Data Characteristics and Stylized Facts
%----------------------------------------------------------------------------------------
% \section{Data Characteristics and Stylized Facts} % Sections are added in order to organize your presentation into discrete blocks, all sections and subsections are automatically output to the table of contents as an overview of the talk but NOT output in the presentation as separate slides
% %----------------------------------------------------------------------------------------
%	NEW CHAPTER TITLE SLIDE
%----------------------------------------------------------------------------------------
\begin{frame}
	\frametitle{Appendix - Figures and Tables}
	\begin{tikzpicture}[remember picture, overlay]
        \node[anchor=north east, xshift=-0.1cm, yshift=+0.1cm] at (current page.north east) {
            \includegraphics[width=1.7cm]{wiso_logo.png}
        };
		\node[anchor=center, xshift=5.5cm, yshift=-1.5cm] at (current page.center) {
            \tikz\node[opacity=0.15] {
                \includegraphics[width=\paperwidth, height=\paperheight, keepaspectratio]{siegel.pdf}
            };
        };
    \end{tikzpicture}
	% \begin{block}{} % Empty block title % Set background color (adjust 'blue!20' as needed)
    %     \centering
    %     {\large \textbf{Theoretical Framework}} % Slightly smaller text
    % \end{block}
	\begin{center}
        \begin{minipage}{0.6\textwidth} % Adjust the width of the block
            \begin{block}{} % Empty block title
                \centering
                {\large \textbf{Data Characteristics and Stylized Facts}} % Slightly smaller text
            \end{block}
        \end{minipage}
    \end{center}
\end{frame}
%----------------------------------------------------------------------------------------
% \subsection{Interest Rate Benchmarks}
%----------------------------------------------------------------------------------------
%	NEW SLIDE
%----------------------------------------------------------------------------------------
\begin{frame}
	\frametitle{Interest Rate Benchmarks - Absolute Levels}
    
    \hypertarget{fig:chap_06_interest_rate_comparison_bis_cbpr_vs_3m_interbank}{}
    \begin{figure}
		\includegraphics[width=0.8\linewidth]{chap_06_interest_rate_comparison_bis_cbpr_vs_3m_interbank.pdf}
		\caption{\tiny Figure: Daily BIS Central Bank Policy Rate and 3M Interbank Rates over the time: 1999 - 2019.\footnote{Own Illustration based on XX, Figure 4, page 5 and data taken from \cite{bis_2025_effective_exchange_rates} and \cite{reitz_2025_applied_econometrics_fx}, last accessed 24.10.25.}\label{fig:chap_06_interest_rate_comparison_bis_cbpr_vs_3m_interbank}}
	\end{figure}
\end{frame}
%----------------------------------------------------------------------------------------
%	NEW SLIDE
%----------------------------------------------------------------------------------------
\begin{frame}
	\frametitle{Interest Rate Benchmarks - Relative Levels}
    
    \hypertarget{fig:chap_06_interest_rate_comparison_bis_cbpr_vs_3m_interbank_diffs}{}
    \begin{figure}
		\includegraphics[width=0.8\linewidth]{chap_06_interest_rate_comparison_bis_cbpr_vs_3m_interbank_diffs.pdf}
		\caption{\tiny Figure: Daily BIS Central Bank Policy Rate and 3M Interbank Rates differentials (USD-EUR) over the time: 1999 - 2019.\footnote{Own Illustration based on XX, Figure 4, page 5 and data taken from \cite{bis_2025_effective_exchange_rates} and \cite{reitz_2025_applied_econometrics_fx}, last accessed 24.10.25.}\label{fig:chap_06_interest_rate_comparison_bis_cbpr_vs_3m_interbank_diffs}}
	\end{figure}
\end{frame}
%----------------------------------------------------------------------------------------
% \subsection{Main variables distributions}
%----------------------------------------------------------------------------------------
%	NEW SLIDE
%----------------------------------------------------------------------------------------
\begin{frame}
	\frametitle{Main variables distributions (raw data - normalized)}
    
    \hypertarget{fig:chap_06_raw_data_normalized_histogram}{}
    \begin{figure}
		\includegraphics[width=0.7\linewidth]{chap_06_raw_data_normalized_histogram.pdf}
		\caption{\tiny Figure: Normalized daily EUR/USD spot exchange rate, oil and gas over the time range: 1999 - 2025.\footnote{Own Illustration based on XX, page X and data taken from \cite{fred_2025_federal_reserve_data}, last accessed 24.10.25.}\label{fig:chap_06_raw_data_normalized_histogram}}
	\end{figure}
\end{frame}
%----------------------------------------------------------------------------------------
%	NEW SLIDE
%----------------------------------------------------------------------------------------
\begin{frame}
	\frametitle{Main variables distributions (log first differences)}
    
    \hypertarget{fig:chap_06_log_first_diff_histogram}{}
    \begin{figure}
		\includegraphics[width=0.7\linewidth]{chap_06_log_first_diff_histogram.pdf}
		\caption{\tiny Figure: Log first differences of daily EUR/USD spot exchange rate, oil and gas over the time range: 1999 - 2025.\footnote{Own Illustration based on XX, page X and data taken from \cite{fred_2025_federal_reserve_data}, last accessed 24.10.25.}\label{fig:chap_06_log_first_diff_histogram}}
	\end{figure}
\end{frame}
%----------------------------------------------------------------------------------------
% \subsection{Tests for Normality}
%----------------------------------------------------------------------------------------
%	NEW SLIDE
%----------------------------------------------------------------------------------------
\begin{frame}
    \frametitle{Tests for Normality (raw data)}

    \hypertarget{tab:chap_06_normality_tests_raw_data}{}
    \scriptsize
    \begin{table}
        \centering
        \resizebox{\textwidth}{!}{
            \begin{tabular}{llrrrrl}
                \toprule
                \textbf{Variable} & \textbf{Test} & \textbf{Statistic} & \textbf{p-value} & \textbf{Significance-level} & \textbf{p-value \textless{} 0.05} & \textbf{Result} \\
                \midrule
                EUR/USD & Shapiro-Wilk & 0.989 & 0.000 & 0.050 & True & Not-Normal \\
                EUR/USD & Kolmogorov-Smirnov & 0.799 & 0.000 & 0.050 & True & Not-Normal \\
                EUR/USD & D'Agostino's $K^2$ & 55.280 & 0.000 & 0.050 & True & Not-Normal \\
                \midrule
                WTI Oil & Shapiro-Wilk & 0.981 & 0.000 & 0.050 & True & Not-Normal \\
                WTI Oil & Kolmogorov-Smirnov & 1.000 & 0.000 & 0.050 & True & Not-Normal \\
                WTI Oil & D'Agostino's $K^2$ & 340.085 & 0.000 & 0.050 & True & Not-Normal \\
                \midrule
                Nat Gas & Shapiro-Wilk & 0.8640 & 0.000 & 0.050 & True & Not-Normal \\
                Nat Gas & Kolmogorov-Smirnov & 0.938 & 0.000 & 0.050 & True & Not-Normal \\
                Nat Gas & D'Agostino's $K^2$ & 2130.008 & 0.000 & 0.050 & True & Not-Normal \\
                \bottomrule
            \end{tabular}
        }
        \caption{\tiny Table: Shapiro-Wilks, Kolmogorov-Smirnov, D'Agostino's $K^2$ test for normality for the EUR/USD spot exchange rate, WTI Oil and Natural Gas daily observations over the time 1999 - 2025.\footnote{Own Illustration based on XX, page X, tests used from: \cite{scipy_2025} and data taken from \cite{fred_2025_federal_reserve_data}, last accessed 24.10.25.}\label{tab:chap_06_normality_tests_raw_data}}
    \end{table}
\end{frame}
%----------------------------------------------------------------------------------------
% \subsection{Tests for Stationarity - ADF Tests}
%----------------------------------------------------------------------------------------
%	NEW SLIDE
%----------------------------------------------------------------------------------------
\begin{frame}
    \frametitle{Tests for Stationarity - ADF Tests (raw data)}

    \hypertarget{tab:chap_06_adf_tests_raw_data}{} % Add a new target for the ADF tests table
    \scriptsize
    \begin{table}
        \centering
        \resizebox{\textwidth}{!}{
        \begin{tabular}{rrlllrll}
            \toprule
            \textbf{ADF Statistic} & \textbf{p-value} & \textbf{Start Time} & \textbf{End Time} & \textbf{Regression Type} & \textbf{Observations} & \textbf{Variable} & \textbf{Result} \\
            \midrule
            -1.847 & 0.357 & 04-01-1999 & 04-01-1999 & c & 6640 & EUR/USD & Non-Stationary \\
            -1.846 & 0.682 & 04-01-1999 & 04-01-1999 & ct & 6640 & EUR/USD & Non-Stationary \\
            -2.655 & 0.480 & 04-01-1999 & 04-01-1999 & ctt & 6640 & EUR/USD & Non-Stationary \\
            -0.254 & 0.594 & 04-01-1999 & 04-01-1999 & n & 6640 & EUR/USD & Non-Stationary \\
            \midrule
            -2.789 & 0.059 & 04-01-1999 & 04-01-1999 & c & 6624 & WTI Oil & Non-Stationary \\
            -2.770 & 0.208 & 04-01-1999 & 04-01-1999 & ct & 6624 & WTI Oil & Non-Stationary \\
            -3.060 & 0.267 & 04-01-1999 & 04-01-1999 & ctt & 6624 & WTI Oil & Non-Stationary \\
            -0.697 & 0.413 & 04-01-1999 & 04-01-1999 & n & 6624 & WTI Oil & Non-Stationary \\
            \midrule
            -4.341 & 0.000 & 04-01-1999 & 04-01-1999 & c & 6633 & Nat Gas & Stationary \\
            -4.743 & 0.001 & 04-01-1999 & 04-01-1999 & ct & 6633 & Nat Gas & Stationary \\
            -4.742 & 0.003 & 04-01-1999 & 04-01-1999 & ctt & 6633 & Nat Gas & Stationary \\
            -1.897 & 0.055 & 04-01-1999 & 04-01-1999 & n & 6633 & Nat Gas & Non-Stationary \\
            \bottomrule
        \end{tabular}
        }
        \caption{\tiny Table: Augmented-Dickey-Fuller (ADF) test for stationarity in various variants for the EUR/USD spot exchange rate, WTI Oil and Natural Gas daily observations over the time 1999 - 2025.\footnote{Own Illustration based on XX, page X, tests used from: \cite{seabold2010statsmodels} and data taken from \cite{fred_2025_federal_reserve_data}, last accessed 24.10.25.}\label{tab:chap_06_adf_tests_raw_data}}
    \end{table}
\end{frame}
%----------------------------------------------------------------------------------------
%	NEW SLIDE
%----------------------------------------------------------------------------------------
\begin{frame}
    \frametitle{Tests for Stationarity - ADF Tests (log first differences)}

    \hypertarget{tab:chap_06_adf_tests_log_first_differences}{} % Add a new target for the ADF tests table
    \scriptsize
    \begin{table}
        \centering
        \resizebox{\textwidth}{!}{
        \begin{tabular}{rrlllrll}
            \toprule
            \textbf{ADF Statistic} & \textbf{p-value} & \textbf{Start Time} & \textbf{End Time} & \textbf{Regression Type} & \textbf{Observations} & \textbf{Variable} & \textbf{Result} \\
            \midrule
            -80.613 & 0.000 & 05-01-1999 & 05-01-1999 & c & 6637 & EUR/USD & Stationary \\
            -80.607 & 0.000 & 05-01-1999 & 05-01-1999 & ct & 6637 & EUR/USD & Stationary \\
            -80.601 & 0.000 & 05-01-1999 & 05-01-1999 & ctt & 6637 & EUR/USD & Stationary \\
            -80.619 & 0.000 & 05-01-1999 & 05-01-1999 & n & 6637 & EUR/USD & Stationary \\
            \midrule
            -14.504 & 0.000 & 05-01-1999 & 05-01-1999 & c & 6603 & WTI Oil & Stationary \\
            -14.525 & 0.000 & 05-01-1999 & 05-01-1999 & ct & 6603 & WTI Oil & Stationary \\
            -14.547 & 0.000 & 05-01-1999 & 05-01-1999 & ctt & 6603 & WTI Oil & Stationary \\
            -14.463 & 0.000 & 05-01-1999 & 05-01-1999 & n & 6603 & WTI Oil & Stationary \\
            \midrule
            -20.094 & 0.000 & 05-01-1999 & 05-01-1999 & c & 6616 & Nat Gas & Stationary \\
            -20.102 & 0.000 & 05-01-1999 & 05-01-1999 & ct & 6616 & Nat Gas & Stationary \\
            -20.121 & 0.000 & 05-01-1999 & 05-01-1999 & ctt & 6616 & Nat Gas & Stationary \\
            -20.095 & 0.000 & 05-01-1999 & 05-01-1999 & n & 6616 & Nat Gas & Stationary \\
            \bottomrule
        \end{tabular}
        }
        \caption{\tiny Table: Augmented-Dickey-Fuller (ADF) test for stationarity in various variants for the EUR/USD spot exchange rate, WTI Oil and Natural Gas daily observations (log first differences) over the time 1999 - 2025.\footnote{Own Illustration based on XX, page X, tests used from: \cite{seabold2010statsmodels} and data taken from \cite{fred_2025_federal_reserve_data}, last accessed 24.10.25.}\label{tab:chap_06_adf_tests_log_first_differences}}
    \end{table}
\end{frame}
%----------------------------------------------------------------------------------------
% \subsection{Tests for Cointegration - Engle and Granger Cointegration Tests}
%----------------------------------------------------------------------------------------
%	NEW SLIDE
%----------------------------------------------------------------------------------------
\begin{frame}
	\frametitle{Tests for Cointegration (raw data)}

    \hypertarget{tab:chap_06_cointegration_test_raw_data}{} % Add a new target for the ADF tests table
    \scriptsize
    \begin{table}
        \centering
        \resizebox{\textwidth}{!}{
        \begin{tabular}{rrllrllll}
        \toprule
        \textbf{Cointegration Score} & \textbf{p-value} & \textbf{Start Time} & \textbf{End Time} & \textbf{Observations} & \textbf{Trend} & \textbf{Variable X} & \textbf{Variable Y} & \textbf{Result} \\
        \midrule
        -2.967000 & 0.118000 & 04-01-1999 & 01-10-2025 & 6641 & c & EUR/USD & WTI Oil & Not Cointegrated \\
        -3.364000 & 0.134000 & 04-01-1999 & 01-10-2025 & 6641 & ct & EUR/USD & WTI Oil & Not Cointegrated \\
        -3.635000 & 0.167000 & 04-01-1999 & 01-10-2025 & 6641 & ctt & EUR/USD & WTI Oil & Not Cointegrated \\
        -3.268000 & 0.013000 & 04-01-1999 & 01-10-2025 & 6641 & n & EUR/USD & WTI Oil & Cointegrated \\
        \midrule
        -2.416000 & 0.317000 & 04-01-1999 & 01-10-2025 & 6641 & c & EUR/USD & Nat Gas & Not Cointegrated \\
        -2.634000 & 0.446000 & 04-01-1999 & 01-10-2025 & 6641 & ct & EUR/USD & Nat Gas & Not Cointegrated \\
        -3.530000 & 0.204000 & 04-01-1999 & 01-10-2025 & 6641 & ctt & EUR/USD & Nat Gas & Not Cointegrated \\
        -4.182000 & 0.001000 & 04-01-1999 & 01-10-2025 & 6641 & n & EUR/USD & Nat Gas & Cointegrated \\
        \bottomrule
        \end{tabular}
        }
        \caption{\tiny Table: Engle and Granger Cointegration test for the EUR/USD spot exchange rate, WTI Oil and Natural Gas daily observations over the time 1999 - 2025.\footnote{Own Illustration based on XX, page X, tests used from: \cite{seabold2010statsmodels} and data taken from \cite{fred_2025_federal_reserve_data}, last accessed 24.10.25.}\label{tab:chap_06_cointegration_test_raw_data}}
    \end{table}
\end{frame}
%----------------------------------------------------------------------------------------
%	NEW SLIDE
%----------------------------------------------------------------------------------------
\begin{frame}
	\frametitle{Tests for Cointegration (log differences)}

    \hypertarget{tab:chap_06_cointegration_test_log_differences}{} % Add a new target for the ADF tests table
    \scriptsize
    \begin{table}
        \centering
        \resizebox{\textwidth}{!}{
        \begin{tabular}{rrllrllll}
        \toprule
        \textbf{Cointegration Score} & \textbf{p-value} & \textbf{Start Time} & \textbf{End Time} & \textbf{Observations} & \textbf{Trend} & \textbf{Variable X} & \textbf{Variable Y} & \textbf{Result} \\
        \midrule
        -4.293000 & 0.003000 & 17-02-1999 & 01-10-2025 & 6609 & c & EUR/USD & WTI Oil & Cointegrated \\
        -4.805000 & 0.002000 & 17-02-1999 & 01-10-2025 & 6609 & ct & EUR/USD & WTI Oil & Cointegrated \\
        -4.830000 & 0.006000 & 17-02-1999 & 01-10-2025 & 6609 & ctt & EUR/USD & WTI Oil & Cointegrated \\
        -5.724000 & 0.000000 & 17-02-1999 & 01-10-2025 & 6609 & n & EUR/USD & WTI Oil & Cointegrated \\
        \midrule
        -4.001000 & 0.007000 & 17-02-1999 & 01-10-2025 & 6609 & c & EUR/USD & Nat Gas & Cointegrated \\
        -4.348000 & 0.009000 & 17-02-1999 & 01-10-2025 & 6609 & ct & EUR/USD & Nat Gas & Cointegrated \\
        -4.341000 & 0.030000 & 17-02-1999 & 01-10-2025 & 6609 & ctt & EUR/USD & Nat Gas & Cointegrated \\
        -4.246000 & 0.000000 & 17-02-1999 & 01-10-2025 & 6609 & n & EUR/USD & Nat Gas & Cointegrated \\
        \bottomrule
    \end{tabular}
    }
    \caption{\tiny Table: Engle and Granger Cointegration test for the EUR/USD spot exchange rate, WTI Oil and Natural Gas daily observations (log first differences) over the time 1999 - 2025.\footnote{Own Illustration based on XX, page X, tests used from: \cite{seabold2010statsmodels} and data taken from \cite{fred_2025_federal_reserve_data}, last accessed 24.10.25.}\label{tab:chap_06_cointegration_test_log_differences}}
    \end{table}
\end{frame}
%----------------------------------------------------------------------------------------
% \subsection{Tests for Autocorrelation - ACF Plot}
%----------------------------------------------------------------------------------------
%	NEW SLIDE
%----------------------------------------------------------------------------------------
\begin{frame}
	\frametitle{Tests for Autocorrelation (raw data)}
    
    \hypertarget{fig:chap_06_acf_plot_raw_series}{}
    \begin{figure}
		\includegraphics[width=0.7\linewidth]{chap_06_acf_plot_raw_series.pdf}
		\caption{\tiny Figure: ACF values for daily observations of the EUR/USD spot exchange rate, oil and gas over the time: 1999 - 2025.\footnote{Own Illustration based on XX, page X, tests used from: \cite{seabold2010statsmodels} and data taken from \cite{fred_2025_federal_reserve_data}, last accessed 24.10.25.}\label{fig:chap_06_acf_plot_raw_series}}
	\end{figure}
\end{frame}
%----------------------------------------------------------------------------------------
%	NEW SLIDE
%----------------------------------------------------------------------------------------
\begin{frame}
	\frametitle{Tests for Autocorrelation (log first differences)}
    
    \hypertarget{fig:chap_06_acf_plot_log_diff}{}
    \begin{figure}
		\includegraphics[width=0.7\linewidth]{chap_06_acf_plot_log_diff.pdf}
		\caption{\tiny Figure: ACF values for daily observations (log first differences) of the EUR/USD spot exchange rate, WTI Oil and Natural Gas over the time: 1999 - 2025.\footnote{Own Illustration based on XX, page X, tests used from: \cite{seabold2010statsmodels} and data taken from \cite{fred_2025_federal_reserve_data}, last accessed 24.10.25.}\label{fig:chap_06_acf_plot_log_diff}}
	\end{figure}
\end{frame}
%----------------------------------------------------------------------------------------
% \subsection{Tests for Partial Autocorrelation - PACF Plot}
%----------------------------------------------------------------------------------------
%	NEW SLIDE
%----------------------------------------------------------------------------------------
\begin{frame}
	\frametitle{Tests for Partial Autocorrelation (raw data)}
    
    \hypertarget{fig:chap_06_pacf_plot_raw_series}{}
    \begin{figure}
		\includegraphics[width=0.7\linewidth]{chap_06_pacf_plot_raw_series.pdf}
		\caption{\tiny Figure: PACF values for daily observations of the EUR/USD spot exchange rate, WTI Oil and Natural Gas over the time: 1999 - 2025.\footnote{Own Illustration based on XX, page X, tests used from: \cite{seabold2010statsmodels} and data taken from \cite{fred_2025_federal_reserve_data}, last accessed 24.10.25.}\label{fig:chap_06_pacf_plot_raw_series}}
	\end{figure}
\end{frame}
%----------------------------------------------------------------------------------------
%	NEW SLIDE
%----------------------------------------------------------------------------------------
\begin{frame}
	\frametitle{Tests for Partial Autocorrelation (log first differences)}
    
    \hypertarget{fig:chap_06_pacf_plot_log_diff}{}
    \begin{figure}
		\includegraphics[width=0.7\linewidth]{chap_06_pacf_plot_log_diff.pdf}
		\caption{\tiny Figure: PACF values for daily observations (log first differences) of the EUR/USD spot exchange rate, WTI Oil and Natural Gas over the time: 1999 - 2025.\footnote{Own Illustration based on XYZ (2000), page 7 and data accessed 10.09.25.}\label{fig:chap_06_pacf_plot_log_diff}}
	\end{figure}
\end{frame}
%----------------------------------------------------------------------------------------
% \subsection{Granger Causality Tests}
%----------------------------------------------------------------------------------------
%	NEW SLIDE
%----------------------------------------------------------------------------------------
\begin{frame}
	\frametitle{Granger Causality Tests - EUR/USD and oil (raw data)}

    \hypertarget{fig:chap_06_granger_causality_test_oil_raw_series}{}
    \begin{figure}
		\includegraphics[width=0.6\linewidth]{chap_06_granger_causality_test_oil_raw_series.pdf}
		\caption{\tiny Figure: Granger causality test results testing granger causality of daily observations of oil for EUR/USD spot exchange rate over the time: 1999 - 2025.\footnote{Own Illustration based on XX, page X, tests used from: \cite{seabold2010statsmodels} and data taken from \cite{fred_2025_federal_reserve_data}, last accessed 24.10.25.}\label{fig:chap_06_granger_causality_test_oil_raw_series}}
	\end{figure}
\end{frame}
%----------------------------------------------------------------------------------------
%	NEW SLIDE
%----------------------------------------------------------------------------------------
\begin{frame}
	\frametitle{Granger Causality Tests - EUR/USD and gas (raw data)}
    
    \hypertarget{fig:chap_06_granger_causality_test_gas_raw_series}{}
    \begin{figure}
		\includegraphics[width=0.6\linewidth]{chap_06_granger_causality_test_gas_raw_series.pdf}
		\caption{\tiny Figure: Granger causality test results testing granger causality of daily observations of gas for EUR/USD over the time: 1999 - 2025.\footnote{Own Illustration based on XX, page X, tests used from: \cite{seabold2010statsmodels} and data taken from \cite{fred_2025_federal_reserve_data}, last accessed 24.10.25.}\label{fig:chap_06_granger_causality_test_gas_raw_series}}
	\end{figure}
\end{frame}
%----------------------------------------------------------------------------------------
%	NEW SLIDE
%----------------------------------------------------------------------------------------
\begin{frame}
	\frametitle{Granger Causality Tests - EUR/USD and oil (log first differences)}

    \hypertarget{fig:chap_06_granger_causality_test_oil_log_diff}{}
    \begin{figure}
		\includegraphics[width=0.6\linewidth]{chap_06_granger_causality_test_oil_log_diff.pdf}
		\caption{\tiny Figure: Granger causality test results testing granger causality of daily observations (log first differences) of WTI Oil for EUR/USD over the time: 1999 - 2025.\footnote{Own Illustration based on XX, page X, tests used from: \cite{seabold2010statsmodels} and data taken from \cite{fred_2025_federal_reserve_data}, last accessed 24.10.25.}\label{fig:chap_06_granger_causality_test_oil_log_diff}}
	\end{figure}
\end{frame}
%----------------------------------------------------------------------------------------
%	NEW SLIDE
%----------------------------------------------------------------------------------------
\begin{frame}
	\frametitle{Granger Causality Tests - EUR/USD and gas (log first differences)}

    \hypertarget{fig:chap_06_granger_causality_test_gas_log_diff}{}
    \begin{figure}
		\includegraphics[width=0.6\linewidth]{chap_06_granger_causality_test_gas_log_diff.pdf}
		\caption{\tiny Figure: Granger causality test results testing granger causality of daily observations (log first differences) of Natural Gas for EUR/USD over the time: 1999 - 2025.\footnote{Own Illustration based on XX, page X, tests used from: \cite{seabold2010statsmodels} and data taken from \cite{fred_2025_federal_reserve_data}, last accessed 24.10.25.}\label{fig:chap_06_granger_causality_test_gas_log_diff}}
	\end{figure}
\end{frame}
%----------------------------------------------------------------------------------------
%	07 - Model Results
%----------------------------------------------------------------------------------------
\section{Model Results} % Sections are added in order to organize your presentation into discrete blocks, all sections and subsections are automatically output to the table of contents as an overview of the talk but NOT output in the presentation as separate slides
%----------------------------------------------------------------------------------------
%	NEW CHAPTER TITLE SLIDE
%----------------------------------------------------------------------------------------
\begin{frame}
	\frametitle{Chapter 7)}
	\begin{tikzpicture}[remember picture, overlay]
        \node[anchor=north east, xshift=-0.1cm, yshift=+0.1cm] at (current page.north east) {
            \includegraphics[width=1.7cm]{wiso_logo.png}
        };
		\node[anchor=center, xshift=5.5cm, yshift=-1.5cm] at (current page.center) {
            \tikz\node[opacity=0.15] {
                \includegraphics[width=\paperwidth, height=\paperheight, keepaspectratio]{siegel.pdf}
            };
        };
    \end{tikzpicture}
	% \begin{block}{} % Empty block title % Set background color (adjust 'blue!20' as needed)
    %     \centering
    %     {\large \textbf{Theoretical Framework}} % Slightly smaller text
    % \end{block}
	\begin{center}
        \begin{minipage}{0.6\textwidth} % Adjust the width of the block
            \begin{block}{} % Empty block title
                \centering
                {\large \textbf{Model Results}} % Slightly smaller text
            \end{block}
        \end{minipage}
    \end{center}
\end{frame}
%----------------------------------------------------------------------------------------
%	08 - Conclusion and Discussion
%----------------------------------------------------------------------------------------
\section{Conclusion and Discussion} % Sections are added in order to organize your presentation into discrete blocks, all sections and subsections are automatically output to the table of contents as an overview of the talk but NOT output in the presentation as separate slides
%----------------------------------------------------------------------------------------
%	NEW CHAPTER TITLE SLIDE
%----------------------------------------------------------------------------------------
\begin{frame}
	\frametitle{Chapter 4)}
	\begin{tikzpicture}[remember picture, overlay]
        \node[anchor=north east, xshift=-0.1cm, yshift=+0.1cm] at (current page.north east) {
            \includegraphics[width=1.7cm]{wiso_logo.png}
        };
		\node[anchor=center, xshift=5.5cm, yshift=-1.5cm] at (current page.center) {
            \tikz\node[opacity=0.15] {
                \includegraphics[width=\paperwidth, height=\paperheight, keepaspectratio]{siegel.pdf}
            };
        };
    \end{tikzpicture}
	% \begin{block}{} % Empty block title % Set background color (adjust 'blue!20' as needed)
    %     \centering
    %     {\large \textbf{Theoretical Framework}} % Slightly smaller text
    % \end{block}
	\begin{center}
        \begin{minipage}{0.6\textwidth} % Adjust the width of the block
            \begin{block}{} % Empty block title
                \centering
                {\large \textbf{Conclusion and Discussion}} % Slightly smaller text
            \end{block}
        \end{minipage}
    \end{center}
\end{frame}
%----------------------------------------------------------------------------------------
\subsection{Seminar Project Summary}
%----------------------------------------------------------------------------------------
%	NEW SLIDE
%----------------------------------------------------------------------------------------
\begin{frame}
	\frametitle{Seminar Project Summary}

    \begin{block}{Main Research Hypothesis}
		Comparing different (configurations of) optimised clustering algorithms to identify (volatility driven) exchange rate regimes in the EUR/USD spot exchange rate yields promising results.
	\end{block}

    \begin{block}{Additional Research Hypothesis I}
		Incorporating commodity prices (WTI Crude Oil Prices and Natural Gas Prices/ their rolling volatilities) as external variables improves the regime identification results.
	\end{block}

    \begin{block}{Additional Research Hypothesis II}
		In comparison with the benchmark models, the clustering techniques seemed to be able to identify regimes more granular, but despite that were not able to yield more reliable/robust results for the standard UIP-relationship.
	\end{block}

\end{frame}
%----------------------------------------------------------------------------------------
\subsection{Seminar Project Limitations}
%----------------------------------------------------------------------------------------
%	NEW SLIDE
%----------------------------------------------------------------------------------------
\begin{frame}
	\frametitle{Seminar Project Limitations}

    \begin{block}{Evaluation Metrics}
		Only used one metric for clustering (Silhouette Score).
	\end{block}

    \begin{block}{Crisis Periods}
		May include more periods or find a suitable solution for overlapping crisis periods.
	\end{block}

    \begin{block}{Hyperparameter Tuning}
		Hyperparameter tuning did not finish in time for all model configurations.
	\end{block}
\end{frame}
%----------------------------------------------------------------------------------------
\subsection{Future Research}
%----------------------------------------------------------------------------------------
%	NEW SLIDE
%----------------------------------------------------------------------------------------
\begin{frame}
	\frametitle{Future Research - Possible Extensions}
    \begin{block}{Other Spot rate pairs}
		Extend the analysis to other spot exchange rate pairs, e.g. focus on the Commodity Currencies with a strong commodity linkage.
	\end{block}
    \begin{block}{Other commodities}
		Extend the analysis to also include other commodities, e.g. Natural Gas, Brent Oil, Gold, etc..
	\end{block}
    \begin{block}{Other macroeconomic/external variables}
		Extend the analysis by incorporating other macroeconomic variables, e.g. Interest Rates, Inflation Rates, etc..
	\end{block}
    \begin{block}{Other data frequencies}
		Extend the analysis by using intraday data to capture more granular dynamics.
	\end{block}
    \begin{block}{Other non-linear models}
		Apply Machine Learning techniques to enhance predictive capabilities and better capture non-linearities.
	\end{block}
\end{frame}
%----------------------------------------------------------------------------------------
%	XX - Appendix
%----------------------------------------------------------------------------------------
\section{Appendix} % Sections are added in order to organize your presentation into discrete blocks, all sections and subsections are automatically output to the table of contents as an overview of the talk but NOT output in the presentation as separate slides
%----------------------------------------------------------------------------------------
%	NEW CHAPTER TITLE SLIDE
%----------------------------------------------------------------------------------------
\begin{frame}
	\frametitle{Chapter X)}
	\begin{tikzpicture}[remember picture, overlay]
        \node[anchor=north east, xshift=-0.1cm, yshift=+0.1cm] at (current page.north east) {
            \includegraphics[width=1.7cm]{wiso_logo.png}
        };
		\node[anchor=center, xshift=5.5cm, yshift=-1.5cm] at (current page.center) {
            \tikz\node[opacity=0.15] {
                \includegraphics[width=\paperwidth, height=\paperheight, keepaspectratio]{siegel.pdf}
            };
        };
    \end{tikzpicture}
	% \begin{block}{} % Empty block title % Set background color (adjust 'blue!20' as needed)
    %     \centering
    %     {\large \textbf{Modern Energy Commodity Markets - The current state}} % Slightly smaller text
    % \end{block}
	\begin{center}
        \begin{minipage}{0.6\textwidth} % Adjust the width of the block
            \begin{block}{} % Empty block title
                \centering
                {\large \textbf{Appendix}} % Slightly smaller text
            \end{block}
        \end{minipage}
    \end{center}
\end{frame}
%----------------------------------------------------------------------------------------
\subsection{Appendix - Figures and Tables}
%----------------------------------------------------------------------------------------
%	NEW SLIDE
%----------------------------------------------------------------------------------------
\begin{frame}
	\frametitle{Appendix - Figures and Tables}
\end{frame}
%----------------------------------------------------------------------------------------
\subsection{Appendix - Data and Definitions}
%----------------------------------------------------------------------------------------
%	NEW SLIDE
%----------------------------------------------------------------------------------------
\begin{frame}
	\frametitle{Appendix - Data and Definitions}
    \begin{itemize}
        \item Nominal effective exchange rate (NEER): Calculated as geometric trade-weighted averages of bilateral exchange rates.\footnote{See: https://data.bis.org/topics/EER}
        \item Real effective exchange rates (REER): Derived by adjusting the NEER by relative consumer prices.\footnote{See: https://data.bis.org/topics/EER}
        \item Nominal exchange rates (NER): The exchange rate between two currencies without adjustment for inflation.\footnote{See: }
        \item Real exchange rates (RER): The nominal exchange rate adjusted for differences in price levels between countries.\footnote{See: }
    \end{itemize}
\end{frame}
%----------------------------------------------------------------------------------------
%	XX - References
%----------------------------------------------------------------------------------------
\section{References} % Sections are added in order to organize your presentation into discrete blocks, all sections and subsections are automatically output to the table of contents as an overview of the talk but NOT output in the presentation as separate slides
%----------------------------------------------------------------------------------------
%	NEW CHAPTER TITLE SLIDE
%----------------------------------------------------------------------------------------
\begin{frame}
	\frametitle{Chapter X)}
	\begin{tikzpicture}[remember picture, overlay]
        \node[anchor=north east, xshift=-0.1cm, yshift=+0.1cm] at (current page.north east) {
            \includegraphics[width=1.7cm]{wiso_logo.png}
        };
		\node[anchor=center, xshift=5.5cm, yshift=-1.5cm] at (current page.center) {
            \tikz\node[opacity=0.15] {
                \includegraphics[width=\paperwidth, height=\paperheight, keepaspectratio]{siegel.pdf}
            };
        };
    \end{tikzpicture}
	% \begin{block}{} % Empty block title % Set background color (adjust 'blue!20' as needed)
    %     \centering
    %     {\large \textbf{References}} % Slightly smaller text
    % \end{block}
	\begin{center}
        \begin{minipage}{0.6\textwidth} % Adjust the width of the block
            \begin{block}{} % Empty block title
                \centering
                {\large \textbf{References}} % Slightly smaller text
            \end{block}
        \end{minipage}
    \end{center}
\end{frame}
%----------------------------------------------------------------------------------------
\subsection{Literature}
%----------------------------------------------------------------------------------------
%	NEW SLIDE
%----------------------------------------------------------------------------------------
\begin{frame} % Use [allowframebreaks] to allow automatic splitting across slides if the content is too long
	\frametitle{References - Literature}
	
	\begin{thebibliography}{99} % Beamer does not support BibTeX so references must be inserted manually as below, you may need to use multiple columns and/or reduce the font size further if you have many references
		\footnotesize % Reduce the font size in the bibliography
		
		\bibitem[Smith, 2022]{p1}
			John Smith (2022)
			\newblock Publication title
			\newblock \emph{Journal Name} 12(3), 45 -- 678.
			
		\bibitem[Kennedy, 2023]{p2}
			Annabelle Kennedy (2023)
			\newblock Publication title
			\newblock \emph{Journal Name} 12(3), 45 -- 678.
	\end{thebibliography}
\end{frame}
%----------------------------------------------------------------------------------------
\subsection{Data}
%----------------------------------------------------------------------------------------
%	NEW SLIDE
%----------------------------------------------------------------------------------------
\begin{frame} % Use [allowframebreaks] to allow automatic splitting across slides if the content is too long
	\frametitle{References - Data}
	
	\begin{thebibliography}{99} % Beamer does not support BibTeX so references must be inserted manually as below, you may need to use multiple columns and/or reduce the font size further if you have many references
		\footnotesize % Reduce the font size in the bibliography
		
		\bibitem[Smith, 2022]{p1}
			John Smith (2022)
			\newblock Publication title
			\newblock \emph{Journal Name} 12(3), 45 -- 678.
			
		\bibitem[Kennedy, 2023]{p2}
			Annabelle Kennedy (2023)
			\newblock Publication title
			\newblock \emph{Journal Name} 12(3), 45 -- 678.
	\end{thebibliography}
\end{frame}
%----------------------------------------------------------------------------------------
%	ACKNOWLEDGMENTS SLIDE
%----------------------------------------------------------------------------------------
% \begin{frame}
	\frametitle{Acknowledgements}
	
	\begin{columns}[t] % The "c" option specifies centered vertical alignment while the "t" option is used for top vertical alignment
		\begin{column}{0.45\textwidth} % Left column width
			\textbf{Smith Lab}
			\begin{itemize}
				\item Alice Smith
				\item Devon Brown
			\end{itemize}
			\textbf{Cook Lab}
			\begin{itemize}
				\item Margaret
				\item Jennifer
				\item Yuan
			\end{itemize}
		\end{column}		
		\begin{column}{0.5\textwidth} % Right column width
			\textbf{Funding}
			\begin{itemize}
				\item British Royal Navy
				\item Norwegian Government
			\end{itemize}
		\end{column}
	\end{columns}
\end{frame}
%----------------------------------------------------------------------------------------
%	LIST OF FIGURES
%----------------------------------------------------------------------------------------
%----------------------------------------------------------------------------------------
%	NEW CHAPTER TITLE SLIDE
%----------------------------------------------------------------------------------------
\begin{frame}
	\frametitle{~}
	\begin{tikzpicture}[remember picture, overlay]
        \node[anchor=north east, xshift=-0.1cm, yshift=+0.1cm] at (current page.north east) {
            \includegraphics[width=1.7cm]{wiso_logo.png}
        };
		\node[anchor=center, xshift=5.5cm, yshift=-1.5cm] at (current page.center) {
            \tikz\node[opacity=0.15] {
                \includegraphics[width=\paperwidth, height=\paperheight, keepaspectratio]{siegel.pdf}
            };
        };
    \end{tikzpicture}
	% \begin{block}{} % Empty block title % Set background color (adjust 'blue!20' as needed)
    %     \centering
    %     {\large \textbf{References}} % Slightly smaller text
    % \end{block}
	\begin{center}
        \begin{minipage}{0.6\textwidth} % Adjust the width of the block
            \begin{block}{} % Empty block title
                \centering
                {\large \textbf{List of Figures}} % Slightly smaller text
            \end{block}
        \end{minipage}
    \end{center}
\end{frame}
%----------------------------------------------------------------------------------------
%	NEW SLIDE
%----------------------------------------------------------------------------------------
\begin{frame}
    \frametitle{List of figures}
    \tiny
	\begin{itemize}
		% Chapter 00
        \listfigure{fig:chap_00_deviations_of_usd_spotrates_from_ppp_values}{Monthly deviations of USD spot rates from PPP values}
		% Chapter 02
		\listfigure{fig:chap_02_us_cpi_inflation_decomposition}{Monthly US CPI: Headline and component contributions over the time: 2001 - 2024}
		\listfigure{fig:chap_02_eu_area_cpi_inflation_decomposition}{Monthly EU Area CPI: Headline and component contributions over the time: 2000 - 2024}
		\listfigure{fig:chap_02_exchange_rate_oil_raw_vola_normalized_crisis_periods_highlighted}{Normalized oil price volatility and EUR/USD exchange rate with crisis periods highlighted}
		% Chapter 03
		\listfigure{fig:chap_03_literature_systematic_overview}{Systematic Literature Overview}
		% Chapter 04
		\listfigure{fig:chap_04_theo_framework_prices_measures}{Theoretical Framework: Price Measures}
		\listfigure{fig:chap_04_theo_framework_simple_model}{Theoretical Framework: Simple Model}
		\listfigure{fig:chap_04_theo_framework_theoretical_framework_I}{Theoretical Framework: Model I}
		\listfigure{fig:chap_04_theo_framework_theoretical_framework_II}{Theoretical Framework: Model II}
		\listfigure{fig:chap_04_theo_framework_theoretical_framework_III}{Theoretical Framework: Model III}
		% Chapter 05
		% \listfigure{}
		% Chapter 07
		\listfigure{fig:chap_07_model_comparison_bar_plot}{Model Comparison Bar Plot}
		\listfigure{fig:chap_07_predicted_model_regimes_with_crisis_periods_highlighted}{Predicted Model Regimes with Crisis Periods Highlighted}
		% Chapter 01
		\listfigure{fig:chap_01_yearly_oil_consumption_production_combined_graph}{Yearly oil consumption and production by country}
		\listfigure{fig:chap_01_yearly_gas_consumption_production_combined_graph}{Yearly natural gas consumption and production by country}
		\listfigure{fig:chap_01_weekly_open_interest_oil_gas_combined_graph}{Weekly open interest in oil and natural gas futures contracts}
		% Chapter 06
		\listfigure{fig:chap_06_interest_rate_comparison_bis_cbpr_vs_3m_interbank}{Daily BIS Central Bank Policy Rate and 3M Interbank Rates over the time: 1999 - 2019}
		\listfigure{fig:chap_06_interest_rate_comparison_bis_cbpr_vs_3m_interbank_diffs}{Daily BIS Central Bank Policy Rate and 3M Interbank Rates differentials (USD-EUR) over the time: 1999 - 2019}
		\listfigure{fig:chap_06_raw_data_normalized_histogram}{Normalized Histograms of Raw Data}
		\listfigure{fig:chap_06_log_first_diff_histogram}{Histograms of Log First Differences}
		\listfigure{fig:chap_06_acf_plot_raw_series}{ACF Plots of Raw Series}
		\listfigure{fig:chap_06_acf_plot_log_diff}{ACF Plots of Log First Differences}
		\listfigure{fig:chap_06_pacf_plot_raw_series}{PACF Plots of Raw Series}
		\listfigure{fig:chap_06_pacf_plot_log_diff}{PACF Plots of Log First Differences}
		\listfigure{fig:chap_06_granger_causality_test_oil_raw_series}{Granger Causality Test Results - Oil Raw Series}
		\listfigure{fig:chap_06_granger_causality_test_gas_raw_series}{Granger Causality Test Results - Gas Raw Series}
		\listfigure{fig:chap_06_granger_causality_test_oil_log_diff}{Granger Causality Test Results - Oil Log First Differences}
		\listfigure{fig:chap_06_granger_causality_test_gas_log_diff}{Granger Causality Test Results - Gas Log First Differences}
		% \listfigure{}
		% \listfigure{}
    \end{itemize}
\end{frame}
%----------------------------------------------------------------------------------------
%	LIST OF TABLES
%----------------------------------------------------------------------------------------
%----------------------------------------------------------------------------------------
%	NEW CHAPTER TITLE SLIDE
%----------------------------------------------------------------------------------------
\begin{frame}
	\frametitle{~}
	\begin{tikzpicture}[remember picture, overlay]
        \node[anchor=north east, xshift=-0.1cm, yshift=+0.1cm] at (current page.north east) {
            \includegraphics[width=1.7cm]{wiso_logo.png}
        };
		\node[anchor=center, xshift=5.5cm, yshift=-1.5cm] at (current page.center) {
            \tikz\node[opacity=0.15] {
                \includegraphics[width=\paperwidth, height=\paperheight, keepaspectratio]{siegel.pdf}
            };
        };
    \end{tikzpicture}
	% \begin{block}{} % Empty block title % Set background color (adjust 'blue!20' as needed)
    %     \centering
    %     {\large \textbf{References}} % Slightly smaller text
    % \end{block}
	\begin{center}
        \begin{minipage}{0.6\textwidth} % Adjust the width of the block
            \begin{block}{} % Empty block title
                \centering
                {\large \textbf{List of Tables}} % Slightly smaller text
            \end{block}
        \end{minipage}
    \end{center}
\end{frame}
%----------------------------------------------------------------------------------------
%	NEW SLIDE
%----------------------------------------------------------------------------------------
\begin{frame}
    \frametitle{List of tables}
    \tiny
    \begin{itemize}
        \listtable{tab:chap_06_normality_tests_raw_data}{Normality Tests for Raw Data}
		\listtable{tab:chap_06_normality_tests_log_first_differences}{Normality Tests for Log First Differences}
        \listtable{tab:chap_06_adf_tests_raw_data}{ADF Tests for Raw Data}
		\listtable{tab:chap_06_adf_tests_log_first_differences}{ADF Tests for Log First Differences}
		\listtable{tab:chap_06_cointegration_test_raw_data}{Cointegration Tests for Raw Data}
		\listtable{tab:chap_06_cointegration_test_log_differences}{Cointegration Tests for Log First Differences}
		\listtable{tab:exemplary_exchange_rates}{Exemplary Exchange Rate Types}
        \listtable{tab:major_global_crisis_periods}{Major Global Crisis Periods}
		% \listtable{}
    \end{itemize}
\end{frame}
%----------------------------------------------------------------------------------------
%	CLOSING SLIDE
%----------------------------------------------------------------------------------------
\begin{frame}%[plain] % The optional argument 'plain' hides the headline and footline
	\frametitle{~}
	\begin{center}
		{\Huge Thank you for your attention!}
		
		\bigskip\bigskip % Vertical whitespace

		{\LARGE We await your Questions and/or Comments.}
		\begin{tikzpicture}[remember picture, overlay]
			\node[anchor=north east, xshift=-0.1cm, yshift=+0.1cm] at (current page.north east) {
				\includegraphics[width=1.7cm]{wiso_logo.png}
			};
			\node[anchor=center, xshift=5.5cm, yshift=-1.5cm] at (current page.center) {
			    \tikz\node[opacity=0.15] {
			        \includegraphics[width=\paperwidth, height=\paperheight, keepaspectratio]{siegel.pdf}
			    };
			};
		\end{tikzpicture}

		\bigskip\bigskip % Vertical whitespace
		\bigskip\bigskip % Vertical whitespace
		\bigskip\bigskip % Vertical whitespace

		\begin{columns}[t] % The "t" option specifies top vertical alignment
			\begin{column}{0.50\textwidth} % Left column width
				\centering % Center the content
				\textbf{Josef Fella} \\ % Name in bold
				{\small\textit{josef.fella@stu.uni-kiel.de}} % Email in italic and smaller font
			\end{column}
			\begin{column}{0.50\textwidth} % Right column width
				\centering % Center the content
				\textbf{Robert Hennings} \\ % Name in bold
				{\small\textit{robert.hennings@stu.uni-kiel.de}} % Email in italic and smaller font
			\end{column}
		\end{columns}
		\bigskip
		\tiny\textit{Public GitHub Project Repository:} \url{https://github.com/RobertHennings/Seminar}
	\end{center}
\end{frame}
%----------------------------------------------------------------------------------------
%	DISCUSSION SLIDE
%----------------------------------------------------------------------------------------
\begin{frame}[plain] % The optional argument 'plain' hides the headline and footline
	\frametitle{Discussion}
	\begin{center}

		\bigskip\bigskip % Vertical whitespace
		\bigskip\bigskip % Vertical whitespace

		\begin{enumerate}
			\item Have you expected this outcome?
			\item What do you think about the dynamics?
			\item What other variables could be potentially included?
		\end{enumerate}

		\begin{tikzpicture}[remember picture, overlay]
			\node[anchor=north east, xshift=-0.1cm, yshift=+0.1cm] at (current page.north east) {
				\includegraphics[width=1.7cm]{wiso_logo.png}
			};
			\node[anchor=center, xshift=5.5cm, yshift=-1.5cm] at (current page.center) {
			    \tikz\node[opacity=0.15] {
			        \includegraphics[width=\paperwidth, height=\paperheight, keepaspectratio]{siegel.pdf}
			    };
			};
		\end{tikzpicture}

		\bigskip\bigskip % Vertical whitespace
		\bigskip\bigskip % Vertical whitespace

		\begin{columns}[t] % The "t" option specifies top vertical alignment
			\begin{column}{0.50\textwidth} % Left column width
				\centering % Center the content
				\textbf{Josef Fella} \\ % Name in bold
				{\small\textit{josef.fella@stu.uni-kiel.de}} % Email in italic and smaller font
			\end{column}
			\begin{column}{0.50\textwidth} % Right column width
				\centering % Center the content
				\textbf{Robert Hennings} \\ % Name in bold
				{\small\textit{robert.hennings@stu.uni-kiel.de}} % Email in italic and smaller font
			\end{column}
		\end{columns}
		\bigskip
		\tiny\textit{Public GitHub Project Repository:} \url{https://github.com/RobertHennings/Seminar}
	\end{center}
\end{frame}
%----------------------------------------------------------------------------------------
%	BACK UP SLIDES
%----------------------------------------------------------------------------------------
\begin{frame}
	\frametitle{Further Material for Illustrations - Questions}
	What is the silhouette score? What are other metrics? Advantages-Disadvantages?
    Why exactly this data?
\end{frame}

%----------------------------------------------------------------------------------------
%	ADDITIONAL EXAMPLES
%----------------------------------------------------------------------------------------
% \subsection{Paragraphs and Lists}

% \begin{frame}
% 	\frametitle{Paragraphs of Text}
	
% 	Sed iaculis \alert{dapibus gravida}. Morbi sed tortor erat, nec interdum arcu. Sed id lorem lectus. Quisque viverra augue id sem ornare non aliquam nibh tristique. Aenean in ligula nisl. Nulla sed tellus ipsum. Donec vestibulum ligula non lorem vulputate fermentum accumsan neque mollis.
	
% 	\bigskip % Vertical whitespace
	
% 	% Quote example
% 	\begin{quote}
% 		Sed diam enim, sagittis nec condimentum sit amet, ullamcorper sit amet libero. Aliquam vel dui orci, a porta odio.\\
% 		--- Someone, somewhere\ldots
% 	\end{quote}
	
% 	\bigskip % Vertical whitespace
	
% 	Nullam id suscipit ipsum. Aenean lobortis commodo sem, ut commodo leo gravida vitae. Pellentesque vehicula ante iaculis arcu pretium rutrum eget sit amet purus. Integer ornare nulla quis neque ultrices lobortis.
% \end{frame}

% %------------------------------------------------

% \begin{frame}
% 	\frametitle{Lists}
% 	\framesubtitle{Bullet Points and Numbered Lists} % Optional subtitle
	
% 	\begin{itemize}
% 		\item Lorem ipsum dolor sit amet, consectetur adipiscing elit
% 		\item Aliquam blandit faucibus nisi, sit amet dapibus enim tempus
% 		\begin{itemize}
% 			\item Lorem ipsum dolor sit amet, consectetur adipiscing elit
% 			\item Nam cursus est eget velit posuere pellentesque
% 		\end{itemize}
% 		\item Nulla commodo, erat quis gravida posuere, elit lacus lobortis est, quis porttitor odio mauris at libero
% 	\end{itemize}
	
% 	\bigskip % Vertical whitespace
	
% 	\begin{enumerate}
% 		\item Nam cursus est eget velit posuere pellentesque
% 		\item Vestibulum faucibus velit a augue condimentum quis convallis nulla gravida 
% 	\end{enumerate}
% \end{frame}

% %------------------------------------------------

% \subsection{Blocks}

% \begin{frame}
% 	\frametitle{Blocks of Highlighted Text}
	
% 	\begin{block}{Block Title}
% 		Lorem ipsum dolor sit amet, consectetur adipiscing elit. Integer lectus nisl, ultricies in feugiat rutrum, porttitor sit amet augue.
% 	\end{block}
	
% 	\begin{exampleblock}{Example Block Title}
% 		Aliquam ut tortor mauris. Sed volutpat ante purus, quis accumsan.
% 	\end{exampleblock}
	
% 	\begin{alertblock}{Alert Block Title}
% 		Pellentesque sed tellus purus. Class aptent taciti sociosqu ad litora torquent per conubia nostra, per inceptos himenaeos.
% 	\end{alertblock}
	
% 	\begin{block}{} % Block without title
% 		Suspendisse tincidunt sagittis gravida. Curabitur condimentum, enim sed venenatis rutrum, ipsum neque consectetur orci.
% 	\end{block}
% \end{frame}

% %------------------------------------------------

% \subsection{Columns}

% \begin{frame}
% 	\frametitle{Multiple Columns}
% 	\framesubtitle{Subtitle} % Optional subtitle
	
% 	\begin{columns}[c] % The "c" option specifies centered vertical alignment while the "t" option is used for top vertical alignment
% 		\begin{column}{0.45\textwidth} % Left column width
% 			\textbf{Heading}
% 			\begin{enumerate}
% 				\item Statement
% 				\item Explanation
% 				\item Example
% 			\end{enumerate}
% 		\end{column}
% 		\begin{column}{0.5\textwidth} % Right column width
% 			Lorem ipsum dolor sit amet, consectetur adipiscing elit. Integer lectus nisl, ultricies in feugiat rutrum, porttitor sit amet augue. Aliquam ut tortor mauris. Sed volutpat ante purus, quis accumsan dolor.
% 		\end{column}
% 	\end{columns}
% \end{frame}

% %------------------------------------------------

% \section{Table and Figure Examples}

% \subsection{Table}

% \begin{frame}
% 	\frametitle{Table}
% 	\framesubtitle{Subtitle} % Optional subtitle
	
% 	\begin{table}
% 		\begin{tabular}{l l l}
% 			\toprule
% 			\textbf{Treatments} & \textbf{Response 1} & \textbf{Response 2}\\
% 			\midrule
% 			Treatment 1 & 0.0003262 & 0.562 \\
% 			Treatment 2 & 0.0015681 & 0.910 \\
% 			Treatment 3 & 0.0009271 & 0.296 \\
% 			\bottomrule
% 		\end{tabular}
% 		\caption{Table caption}
% 	\end{table}
% \end{frame}

% %------------------------------------------------

% \subsection{Figure}

% % \begin{frame}
% % 	\frametitle{Figure}
	
% % 	\begin{figure}
% % 		\includegraphics[width=0.5\linewidth]{problem-01.pdf}
% % 		\caption{Problem - 01.}
% % 	\end{figure}
% % \end{frame}

% %------------------------------------------------

% \section{Mathematics}

% \begin{frame}
% 	\frametitle{Definitions \& Examples}
	
% 	\begin{definition}
% 		A \alert{prime number} is a number that has exactly two divisors.
% 	\end{definition}
	
% 	\smallskip % Vertical whitespace
	
% 	\begin{example}
% 		\begin{itemize}
% 			\item 2 is prime (two divisors: 1 and 2).
% 			\item 3 is prime (two divisors: 1 and 3).
% 			\item 4 is not prime (\alert{three} divisors: 1, 2, and 4).
% 		\end{itemize}
% 	\end{example}
	
% 	\smallskip % Vertical whitespace
	
% 	You can also use the \texttt{theorem}, \texttt{lemma}, \texttt{proof} and \texttt{corollary} environments.
% \end{frame}

% %------------------------------------------------

% \begin{frame}
% 	\frametitle{Theorem, Corollary \& Proof}
	
% 	\begin{theorem}[Mass--energy equivalence]
% 		$E = mc^2$
% 	\end{theorem}
	
% 	\begin{corollary}
% 		$x + y = y + x$
% 	\end{corollary}
	
% 	\begin{proof}
% 		$\omega + \phi = \epsilon$
% 	\end{proof}
% \end{frame}

% %------------------------------------------------

% \begin{frame}
% 	\frametitle{Equation}

% 	\begin{equation}
% 		\cos^3 \theta =\frac{1}{4}\cos\theta+\frac{3}{4}\cos 3\theta
% 	\end{equation}
% \end{frame}

% %------------------------------------------------

% \begin{frame}[fragile] % Need to use the fragile option when verbatim is used in the slide
% 	\frametitle{Verbatim}
	
% 	\begin{example}[Theorem Slide Code]
% 		\begin{verbatim}
% 			\begin{frame}
% 				\frametitle{Theorem}
% 				\begin{theorem}[Mass--energy equivalence]
% 					$E = mc^2$
% 				\end{theorem}
% 		\end{frame}\end{verbatim} % Must be on the same line
% 	\end{example}
% \end{frame}

% %------------------------------------------------

% \begin{frame}
% 	Slide without title.
% \end{frame}

% %------------------------------------------------

% \section{Referencing}

% \begin{frame}
% 	\frametitle{Citing References}
	
% 	An example of the \texttt{\textbackslash cite} command to cite within the presentation:
	
% 	\bigskip % Vertical whitespace
	
% 	This statement requires citation \cite{p1,p2}.
% \end{frame}
%----------------------------------------------------------------------------------------
\end{document} 